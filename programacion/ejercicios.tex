\documentclass[a4paper,12pt]{article}
\usepackage{amsmath, amssymb}
\usepackage{dsfont}
\usepackage{tikz} 
\usepackage[utf8]{inputenc}
%\usepackage[spanish]{babel}
\usepackage{amsmath}
\usepackage{amsfonts}
\usepackage{amssymb}
\usepackage{graphicx}
\usepackage{exercise}
\usepackage{algorithmic}
\usepackage{mathtools}

\author{Pilar Barbero Iriarte}

\newenvironment{exercise}[1]% environment name
{% begin code
  \par\vspace{\baselineskip}\noindent
  \textbf{Ejercicio (#1)}\begin{itshape}%
  \par\vspace{\baselineskip}\noindent\ignorespaces
}%
{% end code
  \end{itshape}\ignorespacesafterend
}




\begin{document}

\begin{titlepage}
\begin{center}


% Upper part of the page. The '~' is needed because \\
% only works if a paragraph has started.

\textsc{\LARGE M\'aster en Modelizaci\'on \\e Investigaci\'on Matem\'atica,\\ Estad\'istica y Computaci\'on }\\[1.5cm]
{\large \today}

\textsc{Ejercicios Programaci\'on Cient\'fica}\\[0.5cm]

% Title
\vfill

{ \huge \bfseries Programaci\'on Cient\'ifica y \'Algebra Computacional \\[0.4cm] }

\vfill



% Author and supervisor
\noindent
\begin{minipage}{0.4\textwidth}
\begin{flushleft} \large
\includegraphics[width=1.1\textwidth]{../images/logoUZ.png}~\\
\emph{Autor:}\\
Pilar Barbero Iriarte 
\end{flushleft}
\end{minipage}%
\begin{minipage}{0.4\textwidth}
\begin{flushright} \large
\includegraphics[width=0.5\textwidth]{../images/logoUNIOVI.png}~\\
%\includegraphics[width=0.5\textwidth]{../images/logoUNIRIOJA.png}~\\[1cm]

\emph{Profesor:} \\
Pedro Alonso Vel\'azquez\\
%Jos\'e Mar\'ia Izquierdo
\end{flushright}
\end{minipage}

% Bottom of the page
\end{center}


\end{titlepage}

\pagebreak
\tableofcontents
\pagebreak

\section{Programaci\'on Cient\'ifica}


\textbf{Reorganizar el proceso de eliminiaci\'on de Neville en t\'erminos de operaciones de nivel 3 de BLAS.}\\

\smallskip

El algoritmo de eliminaci\'on de Neville es una alternativa a la eliminaci\'on Gaussiana que nos permite transformar una matriz cuadrada $n\times n$  en una matriz triangular superior $U$.

Podemos expresar el algoritmo de eliminación de nivel en t\'erminos de operaciones de nivel 2 de la siguiente forma,\\

\begin{algorithmic}[1]\label{neville2}
	\FOR{$i=1 \text{ to } n-1$} 
		\STATE{$ A(n:-1:i+1, i) = A(n:-1:i+1, i)/A(n-1:-1:i, i) $}
		\IF{$i < n$}
			\STATE{$ A(n:-1:i+1, i+1:n) = A(n:-1:i+1, i+1:n) - A(n:-1:i+1, i)\cdot A(n-1:-1:i,i+1:n)$ }
		\ENDIF
	\ENDFOR
\end{algorithmic}

\bigskip

En el paso $i$ del algoritmo, las columnas de la $1$ a la $i$, al igual que las filas de $i+1$ a la $n$ de $U$ ya est\'an computadas. El siguiente paso es computar la fila $i$ de $U$. \\

El algoritmo BLAS de nivel 3 reorganiza las operaciones retrasando el paso 4 del algoritmo de neville de nivel 2 durante \textit{b} pasos. Este n\'umero \textit{b} es un entero peque\~no que se llama \textit{block size}. El valor de \'este par\'ametro var\'ia dependiendo de si queremos maximizar la velocidad del algoritmo, sin embargo, suele ser $b=32$ \'o $64$. M\'as adelante, se aplican todos los pasos n\'umero 4 en una misma multiplicaci\'on de matriz por matriz.\\

Supongamos que ya hemos computado las $i-1$ primeras filas de $U$,


\begin{equation}
\begin{array}{l l}
  		A =
  		\bordermatrix{
  				 	& i-b & b & n-b-i+1 \cr
                	i-1     & A_{11} & A_{12} & A_{13}\\
                	b       & A_{31} & A_{22} & A_{23}\\
                	n-b-i+1 & A_{31} & A_{32} & A_{33}
                }
	  = &
	  \begin{bmatrix}
			L_{11} & 0 & 0 \\
			L_{21} & I & 0 \\
			L_{31} & 0 & I	  	
	  \end{bmatrix}
	  \begin{bmatrix}
			U_{11} & U_{21} & U_{31} \\
			0 & \widetilde{A}_{22} & \widetilde{A}_{23} \\
			0 & \widetilde{A}_{31} & \widetilde{A}_{33}
	  \end{bmatrix}\\
\end{array}
\end{equation}

\bigskip

Ahora, aplicamos el algoritmo de eliminaci\'on de Neville de nivel 2 a la submatriz $\bigl(\begin{smallmatrix}
\widetilde{A}_{22}\\ \widetilde{A}_{32}
\end{smallmatrix} \bigr)$,

\begin{equation}
	\begin{bmatrix}
		\widetilde{A}_{22} \\
		\widetilde{A}_{32}
	\end{bmatrix}
	=
	\begin{bmatrix}
		L_{22} \\
		L_{32}
	\end{bmatrix}
	\cdot
	U_{22}
	=
	\begin{bmatrix}
		L_{22} U_{22} \\
		L_{32} U_{22}
	\end{bmatrix}
\end{equation}

que nos permite reescribir la submatriz de la siguiente forma,

\begin{equation}
\begin{array}{l l}
	\begin{bmatrix}
		\widetilde{A}_{22} & \widetilde{A}_{23} \\
		\widetilde{A}_{32} & \widetilde{A}_{33}
	\end{bmatrix}
	& =
	\begin{bmatrix}
		L_{22} U_{22} & \widetilde{A}_{23} \\
		L_{32} U_{22} & \widetilde{A}_{33}
	\end{bmatrix}\\[.5cm]
	\, & = 
	\begin{bmatrix}
		L_{22} & 0 \\
		L_{32} & I
	\end{bmatrix}
	\cdot
	\begin{bmatrix}
		U_{22} & L_{22}^{-1} \widetilde{A}_{23} \\
		0 & \widetilde{A}_{33} - L_{32}\cdot (L_{22}^{-1} \widetilde{A}_{23})
	\end{bmatrix}\\[.5cm]
	\, & = 
	\begin{bmatrix}
		L_{22} & 0 \\
		L_{32} & I
	\end{bmatrix}
	\cdot
	\begin{bmatrix}
		U_{22} & U_{23} \\
		0 & \widetilde{A}_{33} - L_{32}\cdot U_{23}
	\end{bmatrix}\\[.5cm]
	\, & = 
	\begin{bmatrix}
		L_{22} & 0 \\
		L_{32} & I
	\end{bmatrix}
	\cdot
		\begin{bmatrix}
		U_{22} & U_{23} \\
		0 & \widetilde{\widetilde{A}}_{33}
	\end{bmatrix}\\[.5cm]
\end{array}
\end{equation}

En conjunto, conseguimos una factorizaci\'on $LU$ tal que,

\begin{equation}
	\begin{bmatrix}
		A_{11} & A_{12} & A_{13} \\
		A_{21} & A_{22} & A_{23} \\
		A_{31} & A_{32} & A_{33} 
	\end{bmatrix}
	=
	\begin{bmatrix}
		L_{11} & 0 & 0 \\
		L_{21} & L_{22} & 0 \\
		L_{31} & L_{32} & I 
	\end{bmatrix}
	\cdot
	\begin{bmatrix}
		U_{11} & U_{21} & U_{31} \\
		0 & U_{22} & U_{23} \\
		0 & 0 & \widetilde{\widetilde{A}}_{33} 
	\end{bmatrix}
\end{equation}

Esto definde un algoritmo con los siguientes tres pasos, 

\begin{itemize}
	\item Usar el algoritmo de nivel 2 de Neville para factorizar,
	$$\begin{pmatrix} \widetilde{A}_{22}\\ \widetilde{A}_{32} 
\end{pmatrix} = \begin{pmatrix} L_{22} \\ L{32} \end{pmatrix} \cdot U_{22}$$
	\item Asignar $U_{23} = L_{22}^{-1} \widetilde{A}_{23}$. Esto implica resolver un sistema triangular lineal utilizando una operaci\'on simple BLAS.
	\item Asignar $\widetilde{\widetilde{A}}_{33} = \widetilde{A}_{33} - L_{32}\cdot U_{23}$, una multiplicaci\'on matriz por matriz.
\end{itemize}

\bigskip

Formalmente,

\begin{algorithmic}[1]\label{neville3}
	\FOR{$i=1:b:n-1$ }
		\STATE{Factorizar $A(i:n, i:i+b-1) = \bigl(\begin{smallmatrix} L_{22} \\ L_{32} \end{smallmatrix}\bigr) U_{22}$
		}
		\STATE{$A(i:i+b-1, i+b:n) = L_{22}^{-1}\cdot A(i:i+b-1, i+b:n)$ /* Aqu\'i se asigna $U_{23}$ */}
		\STATE{$A(i+b:n, i+b:n) = A(i+b:n, i+b:n) - A(i+b:n, i:i+b-1)\cdot A(i:i+b-1, i+b:n)$ /* Aqu\'i se asigna $\widetilde{\widetilde{A}}_{33}$ */} 
	\ENDFOR
\end{algorithmic}

\bigskip

Como hemos comentado antes, se necesita asignar un valor al par\'ametro $b$ con el fin de maximizar la velocidad del algoritmo. Por un lado, se prefiere un $b$ grande porque se puede ver que la velocidad aumenta cuando se multiplican matrices de gran tama\~no. Por otro lado, se puede verificar que el n\'umero de operaciones de coma flotante realizadas en la l\'inea 1 de nuestro algoritmo es de $n^2 b/2$ por lo que no interesa que $b$ sea un n\'umero grande. \'Este valor suele tomarse seg\'un la m\'aquina en la que se realicen los c\'alculos y suele ser $b=32$ \'o $b=64$.

	  
%%%%%%%%%%%%%%%%%%%%%%%%%%%%%%%%%%%%%%%%%%%%%%%%%%%%%%%%%%%%%%%%%%%%%%%%
%%%%%%%%%%%%%%%%%%%%%%%%%%%%%%%%%%%%%%%%%%%%%%%%%%%%%%%%%%%%%%%%%%%%%%%%\\
%%%%%%%%%%%%%%%%%%%%%%%%%%%%%%%%%%%%%%%%%%%%%%%%%%%%%%%%%%%%%%%%%%%%%%%%\\

%\section{\'Algebra Computacional}
%
%\begin{exercise}{1}
%Determina usando t\'ecnicas de Groebner si los siguientes ideales son iguales:
%
%\begin{itemize}
%	\item $< y^3 - z^2, xz - y^2, xy - z, x^2 -y >$
%	\item $< xy - z^2, xz - y^2, xy - z, x^2 - y >$
%	\item $< xz - y^2, x + y^2 - z - 1, xyz - 1>$
%	\item $< y^2 -x^2y, z - xy, y - x^2 >$
%\end{itemize}
%
%Puedes ayudarte del ordenador.
%
%\end{exercise}
%
%\begin{exercise}{2}
%
%Calcula, sin usar el ordenador, mediante el algoritmo de Buchberger una base reducida del ideal $I = <xy + z, x^2+y^2>$. Determina tambi\'en sin usar el ordenador si la clase $[x + 1]$ es invertible en $k[x,y,z]/I$
%
%\end{exercise}
%
%\begin{exercise}{3}
%
%Calcula el abanico de Groebner del ideal $<x^2 - y^3, x^3 - y^2 + x>$.
%
%\end{exercise}
%
%\begin{exercise}{4}
%
%¿Puede escribirse $4x^4y^2 + 4y^6 - 2x^4 - 4x^2y^2 - 6y^4 + 2x^2 + 4y^2 -1$ de la forma $h(x^2 + y^2 - 1, x^2 - y^2)$ para alg\'un polinomio $h\in \mathbb{Q}[x,y]$?
%
%\end{exercise}

\end{document}