\documentclass[a4paper,12pt]{article}
\usepackage{amsmath, amssymb}

\usepackage[utf8]{inputenc}
%\usepackage[spanish]{babel}
\usepackage{amsmath}
\usepackage{amsfonts}
\usepackage{amssymb}
\usepackage{graphicx}
\usepackage{exercise}

\author{Pilar Barbero Iriarte}

\newenvironment{exercise}[1]% environment name
{% begin code
  \par\vspace{\baselineskip}\noindent
  \textbf{Ejercicio (#1)}\begin{itshape}%
  \par\vspace{\baselineskip}\noindent\ignorespaces
}%
{% end code
  \end{itshape}\ignorespacesafterend
}


\begin{document}

\begin{titlepage}
\begin{center}


% Upper part of the page. The '~' is needed because \\
% only works if a paragraph has started.

\textsc{\LARGE M\'aster en Modelizaci\'on \\e Investigaci\'on Matem\'atica,\\ Estad\'istica y Computaci\'on }\\[1.5cm]
{\large \today}

\textsc{Ejercicios Primera Parte}\\[0.5cm]

% Title
\vfill

{ \huge \bfseries Modelos de Log\'istica \\[0.4cm] }

\vfill



% Author and supervisor
\noindent
\begin{minipage}{0.4\textwidth}
\begin{flushleft} \large
\includegraphics[width=1.1\textwidth]{../logoUZ.png}~\\[1cm]
\emph{Autor:}\\
Pilar Barbero Iriarte 
\end{flushleft}
\end{minipage}%
\begin{minipage}{0.4\textwidth}
\begin{flushright} \large
\includegraphics[width=0.85\textwidth]{../logoULL.png}~\\[1cm]
\emph{Profesor:} \\
Antonio Sede\~no Noda
\end{flushright}
\end{minipage}

% Bottom of the page
\end{center}


\end{titlepage}

\pagebreak
\tableofcontents
\pagebreak

\section{Modelizaci\'on}

\begin{exercise}{1}
Un fabricante de tecnolog\'ia solar considera producir masa de m\'odulo de celda (producto 1, $x_1$) para sat\'elites de aplicaci\'on atmosf\'erica  y m\'odulo de celda de astilla (producto 2, $x_2$) para uso de calculadoras de bolsillo. Se conoce una lista donde las ventas son estimadas entre $5$ y $12$ unidades para los productos $1$ y $2$, respectivamente, como pron\'ostico del dpto. de marketing.

El presidente de la empresa ha establecido las siguiente meta para esta particular operaci\'on de producci\'on para el siguiente mes:

Tener un beneficio de al menos 33 unidades, es decir, $x_1 + 3x_2 \geq 33$\\
Objetivo: Limitar el l\'imite de roturas a 36, es decir, $3x_1 + 2x_2 \leq 36$\\
Las restricciones asociadas al proceso de producci\'on son:\\

\begin{itemize}

\item Capacidad de la m\'aquina: $0.5x_1 + 0.25x_2 \leq 8$

\item Capacidad de ensamblado: $0.2x_1 + 0.2x_2 \leq 4$

\item Materiales bruto: $x_1 + 5x_2 \leq 72$

\end{itemize}
\end{exercise}


Modelamos de la siguiente forma:


\begin{itemize}

\item[\textbf{Variables}]
	\begin{itemize}
		\item[] $x_1 \geq 5$
		\item[] $x_2 \geq 12$
	\end{itemize}
	
\item[\textbf{Restricciones}]

\begin{itemize}

		\item[] $3x_1 + 2x_2 \leq 36$
		\item[] $0.5x_1 + 0.25x_1 \leq 8$
		\item[] $0.2x_1 + 0.2x_2 \leq 4$
		\item[] $x_1 + 5x_2 \leq 72$
		\item[] $x_1 + 3x_2 \geq 33$


\end{itemize}
	
	
\item[\textbf{Funci\'on Objetivo}]
	\begin{itemize}
			\item[] $\text{m\'in}\,z = 33 - x_1 - 3x_2$
	\end{itemize}
\end{itemize}



\end{document}