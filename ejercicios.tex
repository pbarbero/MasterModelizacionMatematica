\documentclass[a4paper,12pt]{article}
\usepackage{amsmath, amssymb}
\usepackage{dsfont}
\usepackage{tikz} 
\usepackage[utf8]{inputenc}
%\usepackage[spanish]{babel}
\usepackage{amsmath}
\usepackage{amsfonts}
\usepackage{amssymb}
\usepackage{graphicx}
\usepackage{exercise}

\author{Pilar Barbero Iriarte}

\newenvironment{exercise}[1]% environment name
{% begin code
  \par\vspace{\baselineskip}\noindent
  \textbf{Ejercicio (#1)}\begin{itshape}%
  \par\vspace{\baselineskip}\noindent\ignorespaces
}%
{% end code
  \end{itshape}\ignorespacesafterend
}




\begin{document}

\begin{titlepage}
\begin{center}


% Upper part of the page. The '~' is needed because \\
% only works if a paragraph has started.

\textsc{\LARGE M\'aster en Modelizaci\'on \\e Investigaci\'on Matem\'atica,\\ Estad\'istica y Computaci\'on }\\[1.5cm]
{\large \today}

%\textsc{Ejercicios}\\[0.5cm]

% Title
\vfill

{ \huge \bfseries Programaci\'on Cient\'ifica y \'Algebra Computacional \\[0.4cm] }

\vfill



% Author and supervisor
\noindent
\begin{minipage}{0.4\textwidth}
\begin{flushleft} \large
\includegraphics[width=1.1\textwidth]{images/logoUZ.png}~\\[4.5cm]
\emph{Autor:}\\
Pilar Barbero Iriarte 
\end{flushleft}
\end{minipage}%
\begin{minipage}{0.4\textwidth}
\begin{flushright} \large
\includegraphics[width=0.5\textwidth]{images/logoUNIOVI.png}~\\[1cm]
\includegraphics[width=0.5\textwidth]{images/logoUNIRIOJA.png}~\\[1cm]

\emph{Profesores:} \\
Pedro Alonso Vel\'azquez\\
Jos\'e Mar\'ia Izquierdo
\end{flushright}
\end{minipage}

% Bottom of the page
\end{center}


\end{titlepage}

\pagebreak
\tableofcontents
\pagebreak

\section{Programaci\'on Cient\'ifica}

\begin{exercise}{1}

Reorganizar el proceso de eliminiaci\'on de Neville en t\'erminos de operaciones de nivel 3 de BLAS.

\end{exercise}

\section{\'Algebra Computacional}

\begin{exercise}{1}
Determina usando t\'ecnicas de Groebner si los siguientes ideales son iguales:

\begin{itemize}
	\item $< y^3 - z^2, xz - y^2, xy - z, x^2 -y >$
	\item $< xy - z^2, xz - y^2, xy - z, x^2 - y >$
	\item $< xz - y^2, x + y^2 - z - 1, xyz - 1>$
	\item $< y^2 -x^2y, z - xy, y - x^2 >$
\end{itemize}

Puedes ayudarte del ordenador.

\end{exercise}

\begin{exercise}{2}

Calcula, sin usar el ordenador, mediante el algoritmo de Buchberger una base reducida del ideal $I = <xy + z, x^2+y^2>$. Determina tambi\'en sin usar el ordenador si la clase $[x + 1]$ es invertible en $k[x,y,z]/I$

\end{exercise}

\begin{exercise}{3}

Calcula el abanico de Groebner del ideal $<x^2 - y^3, x^3 - y^2 + x>$.

\end{exercise}

\begin{exercise}{4}

¿Puede escribirse $4x^4y^2 + 4y^6 - 2x^4 - 4x^2y^2 - 6y^4 + 2x^2 + 4y^2 -1$ de la forma $h(x^2 + y^2 - 1, x^2 - y^2)$ para alg\'un polinomio $h\in \mathbb{Q}[x,y]$?

\end{exercise}

\end{document}