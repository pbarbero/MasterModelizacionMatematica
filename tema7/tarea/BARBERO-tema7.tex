
% Default to the notebook output style

    


% Inherit from the specified cell style.




    
\documentclass{article}

    
    
    \usepackage{graphicx} % Used to insert images
    \usepackage{adjustbox} % Used to constrain images to a maximum size 
    \usepackage{color} % Allow colors to be defined
    \usepackage{enumerate} % Needed for markdown enumerations to work
    \usepackage{geometry} % Used to adjust the document margins
    \usepackage{amsmath} % Equations
    \usepackage{amssymb} % Equations
    \usepackage[mathletters]{ucs} % Extended unicode (utf-8) support
    \usepackage[utf8x]{inputenc} % Allow utf-8 characters in the tex document
    \usepackage{fancyvrb} % verbatim replacement that allows latex
    \usepackage{grffile} % extends the file name processing of package graphics 
                         % to support a larger range 
    % The hyperref package gives us a pdf with properly built
    % internal navigation ('pdf bookmarks' for the table of contents,
    % internal cross-reference links, web links for URLs, etc.)
    \usepackage{hyperref}
    \usepackage{longtable} % longtable support required by pandoc >1.10
    \usepackage{booktabs}  % table support for pandoc > 1.12.2
    

    
    
    \definecolor{orange}{cmyk}{0,0.4,0.8,0.2}
    \definecolor{darkorange}{rgb}{.71,0.21,0.01}
    \definecolor{darkgreen}{rgb}{.12,.54,.11}
    \definecolor{myteal}{rgb}{.26, .44, .56}
    \definecolor{gray}{gray}{0.45}
    \definecolor{lightgray}{gray}{.95}
    \definecolor{mediumgray}{gray}{.8}
    \definecolor{inputbackground}{rgb}{.95, .95, .85}
    \definecolor{outputbackground}{rgb}{.95, .95, .95}
    \definecolor{traceback}{rgb}{1, .95, .95}
    % ansi colors
    \definecolor{red}{rgb}{.6,0,0}
    \definecolor{green}{rgb}{0,.65,0}
    \definecolor{brown}{rgb}{0.6,0.6,0}
    \definecolor{blue}{rgb}{0,.145,.698}
    \definecolor{purple}{rgb}{.698,.145,.698}
    \definecolor{cyan}{rgb}{0,.698,.698}
    \definecolor{lightgray}{gray}{0.5}
    
    % bright ansi colors
    \definecolor{darkgray}{gray}{0.25}
    \definecolor{lightred}{rgb}{1.0,0.39,0.28}
    \definecolor{lightgreen}{rgb}{0.48,0.99,0.0}
    \definecolor{lightblue}{rgb}{0.53,0.81,0.92}
    \definecolor{lightpurple}{rgb}{0.87,0.63,0.87}
    \definecolor{lightcyan}{rgb}{0.5,1.0,0.83}
    
    % commands and environments needed by pandoc snippets
    % extracted from the output of `pandoc -s`
    \DefineVerbatimEnvironment{Highlighting}{Verbatim}{commandchars=\\\{\}}
    % Add ',fontsize=\small' for more characters per line
    \newenvironment{Shaded}{}{}
    \newcommand{\KeywordTok}[1]{\textcolor[rgb]{0.00,0.44,0.13}{\textbf{{#1}}}}
    \newcommand{\DataTypeTok}[1]{\textcolor[rgb]{0.56,0.13,0.00}{{#1}}}
    \newcommand{\DecValTok}[1]{\textcolor[rgb]{0.25,0.63,0.44}{{#1}}}
    \newcommand{\BaseNTok}[1]{\textcolor[rgb]{0.25,0.63,0.44}{{#1}}}
    \newcommand{\FloatTok}[1]{\textcolor[rgb]{0.25,0.63,0.44}{{#1}}}
    \newcommand{\CharTok}[1]{\textcolor[rgb]{0.25,0.44,0.63}{{#1}}}
    \newcommand{\StringTok}[1]{\textcolor[rgb]{0.25,0.44,0.63}{{#1}}}
    \newcommand{\CommentTok}[1]{\textcolor[rgb]{0.38,0.63,0.69}{\textit{{#1}}}}
    \newcommand{\OtherTok}[1]{\textcolor[rgb]{0.00,0.44,0.13}{{#1}}}
    \newcommand{\AlertTok}[1]{\textcolor[rgb]{1.00,0.00,0.00}{\textbf{{#1}}}}
    \newcommand{\FunctionTok}[1]{\textcolor[rgb]{0.02,0.16,0.49}{{#1}}}
    \newcommand{\RegionMarkerTok}[1]{{#1}}
    \newcommand{\ErrorTok}[1]{\textcolor[rgb]{1.00,0.00,0.00}{\textbf{{#1}}}}
    \newcommand{\NormalTok}[1]{{#1}}
    
    % Define a nice break command that doesn't care if a line doesn't already
    % exist.
    \def\br{\hspace*{\fill} \\* }
    % Math Jax compatability definitions
    \def\gt{>}
    \def\lt{<}
    % Document parameters
    \title{BARBERO-tema7}
    
    
    

    % Pygments definitions
    
\makeatletter
\def\PY@reset{\let\PY@it=\relax \let\PY@bf=\relax%
    \let\PY@ul=\relax \let\PY@tc=\relax%
    \let\PY@bc=\relax \let\PY@ff=\relax}
\def\PY@tok#1{\csname PY@tok@#1\endcsname}
\def\PY@toks#1+{\ifx\relax#1\empty\else%
    \PY@tok{#1}\expandafter\PY@toks\fi}
\def\PY@do#1{\PY@bc{\PY@tc{\PY@ul{%
    \PY@it{\PY@bf{\PY@ff{#1}}}}}}}
\def\PY#1#2{\PY@reset\PY@toks#1+\relax+\PY@do{#2}}

\expandafter\def\csname PY@tok@gd\endcsname{\def\PY@tc##1{\textcolor[rgb]{0.63,0.00,0.00}{##1}}}
\expandafter\def\csname PY@tok@gu\endcsname{\let\PY@bf=\textbf\def\PY@tc##1{\textcolor[rgb]{0.50,0.00,0.50}{##1}}}
\expandafter\def\csname PY@tok@gt\endcsname{\def\PY@tc##1{\textcolor[rgb]{0.00,0.27,0.87}{##1}}}
\expandafter\def\csname PY@tok@gs\endcsname{\let\PY@bf=\textbf}
\expandafter\def\csname PY@tok@gr\endcsname{\def\PY@tc##1{\textcolor[rgb]{1.00,0.00,0.00}{##1}}}
\expandafter\def\csname PY@tok@cm\endcsname{\let\PY@it=\textit\def\PY@tc##1{\textcolor[rgb]{0.25,0.50,0.50}{##1}}}
\expandafter\def\csname PY@tok@vg\endcsname{\def\PY@tc##1{\textcolor[rgb]{0.10,0.09,0.49}{##1}}}
\expandafter\def\csname PY@tok@m\endcsname{\def\PY@tc##1{\textcolor[rgb]{0.40,0.40,0.40}{##1}}}
\expandafter\def\csname PY@tok@mh\endcsname{\def\PY@tc##1{\textcolor[rgb]{0.40,0.40,0.40}{##1}}}
\expandafter\def\csname PY@tok@go\endcsname{\def\PY@tc##1{\textcolor[rgb]{0.53,0.53,0.53}{##1}}}
\expandafter\def\csname PY@tok@ge\endcsname{\let\PY@it=\textit}
\expandafter\def\csname PY@tok@vc\endcsname{\def\PY@tc##1{\textcolor[rgb]{0.10,0.09,0.49}{##1}}}
\expandafter\def\csname PY@tok@il\endcsname{\def\PY@tc##1{\textcolor[rgb]{0.40,0.40,0.40}{##1}}}
\expandafter\def\csname PY@tok@cs\endcsname{\let\PY@it=\textit\def\PY@tc##1{\textcolor[rgb]{0.25,0.50,0.50}{##1}}}
\expandafter\def\csname PY@tok@cp\endcsname{\def\PY@tc##1{\textcolor[rgb]{0.74,0.48,0.00}{##1}}}
\expandafter\def\csname PY@tok@gi\endcsname{\def\PY@tc##1{\textcolor[rgb]{0.00,0.63,0.00}{##1}}}
\expandafter\def\csname PY@tok@gh\endcsname{\let\PY@bf=\textbf\def\PY@tc##1{\textcolor[rgb]{0.00,0.00,0.50}{##1}}}
\expandafter\def\csname PY@tok@ni\endcsname{\let\PY@bf=\textbf\def\PY@tc##1{\textcolor[rgb]{0.60,0.60,0.60}{##1}}}
\expandafter\def\csname PY@tok@nl\endcsname{\def\PY@tc##1{\textcolor[rgb]{0.63,0.63,0.00}{##1}}}
\expandafter\def\csname PY@tok@nn\endcsname{\let\PY@bf=\textbf\def\PY@tc##1{\textcolor[rgb]{0.00,0.00,1.00}{##1}}}
\expandafter\def\csname PY@tok@no\endcsname{\def\PY@tc##1{\textcolor[rgb]{0.53,0.00,0.00}{##1}}}
\expandafter\def\csname PY@tok@na\endcsname{\def\PY@tc##1{\textcolor[rgb]{0.49,0.56,0.16}{##1}}}
\expandafter\def\csname PY@tok@nb\endcsname{\def\PY@tc##1{\textcolor[rgb]{0.00,0.50,0.00}{##1}}}
\expandafter\def\csname PY@tok@nc\endcsname{\let\PY@bf=\textbf\def\PY@tc##1{\textcolor[rgb]{0.00,0.00,1.00}{##1}}}
\expandafter\def\csname PY@tok@nd\endcsname{\def\PY@tc##1{\textcolor[rgb]{0.67,0.13,1.00}{##1}}}
\expandafter\def\csname PY@tok@ne\endcsname{\let\PY@bf=\textbf\def\PY@tc##1{\textcolor[rgb]{0.82,0.25,0.23}{##1}}}
\expandafter\def\csname PY@tok@nf\endcsname{\def\PY@tc##1{\textcolor[rgb]{0.00,0.00,1.00}{##1}}}
\expandafter\def\csname PY@tok@si\endcsname{\let\PY@bf=\textbf\def\PY@tc##1{\textcolor[rgb]{0.73,0.40,0.53}{##1}}}
\expandafter\def\csname PY@tok@s2\endcsname{\def\PY@tc##1{\textcolor[rgb]{0.73,0.13,0.13}{##1}}}
\expandafter\def\csname PY@tok@vi\endcsname{\def\PY@tc##1{\textcolor[rgb]{0.10,0.09,0.49}{##1}}}
\expandafter\def\csname PY@tok@nt\endcsname{\let\PY@bf=\textbf\def\PY@tc##1{\textcolor[rgb]{0.00,0.50,0.00}{##1}}}
\expandafter\def\csname PY@tok@nv\endcsname{\def\PY@tc##1{\textcolor[rgb]{0.10,0.09,0.49}{##1}}}
\expandafter\def\csname PY@tok@s1\endcsname{\def\PY@tc##1{\textcolor[rgb]{0.73,0.13,0.13}{##1}}}
\expandafter\def\csname PY@tok@sh\endcsname{\def\PY@tc##1{\textcolor[rgb]{0.73,0.13,0.13}{##1}}}
\expandafter\def\csname PY@tok@sc\endcsname{\def\PY@tc##1{\textcolor[rgb]{0.73,0.13,0.13}{##1}}}
\expandafter\def\csname PY@tok@sx\endcsname{\def\PY@tc##1{\textcolor[rgb]{0.00,0.50,0.00}{##1}}}
\expandafter\def\csname PY@tok@bp\endcsname{\def\PY@tc##1{\textcolor[rgb]{0.00,0.50,0.00}{##1}}}
\expandafter\def\csname PY@tok@c1\endcsname{\let\PY@it=\textit\def\PY@tc##1{\textcolor[rgb]{0.25,0.50,0.50}{##1}}}
\expandafter\def\csname PY@tok@kc\endcsname{\let\PY@bf=\textbf\def\PY@tc##1{\textcolor[rgb]{0.00,0.50,0.00}{##1}}}
\expandafter\def\csname PY@tok@c\endcsname{\let\PY@it=\textit\def\PY@tc##1{\textcolor[rgb]{0.25,0.50,0.50}{##1}}}
\expandafter\def\csname PY@tok@mf\endcsname{\def\PY@tc##1{\textcolor[rgb]{0.40,0.40,0.40}{##1}}}
\expandafter\def\csname PY@tok@err\endcsname{\def\PY@bc##1{\setlength{\fboxsep}{0pt}\fcolorbox[rgb]{1.00,0.00,0.00}{1,1,1}{\strut ##1}}}
\expandafter\def\csname PY@tok@kd\endcsname{\let\PY@bf=\textbf\def\PY@tc##1{\textcolor[rgb]{0.00,0.50,0.00}{##1}}}
\expandafter\def\csname PY@tok@ss\endcsname{\def\PY@tc##1{\textcolor[rgb]{0.10,0.09,0.49}{##1}}}
\expandafter\def\csname PY@tok@sr\endcsname{\def\PY@tc##1{\textcolor[rgb]{0.73,0.40,0.53}{##1}}}
\expandafter\def\csname PY@tok@mo\endcsname{\def\PY@tc##1{\textcolor[rgb]{0.40,0.40,0.40}{##1}}}
\expandafter\def\csname PY@tok@kn\endcsname{\let\PY@bf=\textbf\def\PY@tc##1{\textcolor[rgb]{0.00,0.50,0.00}{##1}}}
\expandafter\def\csname PY@tok@mi\endcsname{\def\PY@tc##1{\textcolor[rgb]{0.40,0.40,0.40}{##1}}}
\expandafter\def\csname PY@tok@gp\endcsname{\let\PY@bf=\textbf\def\PY@tc##1{\textcolor[rgb]{0.00,0.00,0.50}{##1}}}
\expandafter\def\csname PY@tok@o\endcsname{\def\PY@tc##1{\textcolor[rgb]{0.40,0.40,0.40}{##1}}}
\expandafter\def\csname PY@tok@kr\endcsname{\let\PY@bf=\textbf\def\PY@tc##1{\textcolor[rgb]{0.00,0.50,0.00}{##1}}}
\expandafter\def\csname PY@tok@s\endcsname{\def\PY@tc##1{\textcolor[rgb]{0.73,0.13,0.13}{##1}}}
\expandafter\def\csname PY@tok@kp\endcsname{\def\PY@tc##1{\textcolor[rgb]{0.00,0.50,0.00}{##1}}}
\expandafter\def\csname PY@tok@w\endcsname{\def\PY@tc##1{\textcolor[rgb]{0.73,0.73,0.73}{##1}}}
\expandafter\def\csname PY@tok@kt\endcsname{\def\PY@tc##1{\textcolor[rgb]{0.69,0.00,0.25}{##1}}}
\expandafter\def\csname PY@tok@ow\endcsname{\let\PY@bf=\textbf\def\PY@tc##1{\textcolor[rgb]{0.67,0.13,1.00}{##1}}}
\expandafter\def\csname PY@tok@sb\endcsname{\def\PY@tc##1{\textcolor[rgb]{0.73,0.13,0.13}{##1}}}
\expandafter\def\csname PY@tok@k\endcsname{\let\PY@bf=\textbf\def\PY@tc##1{\textcolor[rgb]{0.00,0.50,0.00}{##1}}}
\expandafter\def\csname PY@tok@se\endcsname{\let\PY@bf=\textbf\def\PY@tc##1{\textcolor[rgb]{0.73,0.40,0.13}{##1}}}
\expandafter\def\csname PY@tok@sd\endcsname{\let\PY@it=\textit\def\PY@tc##1{\textcolor[rgb]{0.73,0.13,0.13}{##1}}}

\def\PYZbs{\char`\\}
\def\PYZus{\char`\_}
\def\PYZob{\char`\{}
\def\PYZcb{\char`\}}
\def\PYZca{\char`\^}
\def\PYZam{\char`\&}
\def\PYZlt{\char`\<}
\def\PYZgt{\char`\>}
\def\PYZsh{\char`\#}
\def\PYZpc{\char`\%}
\def\PYZdl{\char`\$}
\def\PYZhy{\char`\-}
\def\PYZsq{\char`\'}
\def\PYZdq{\char`\"}
\def\PYZti{\char`\~}
% for compatibility with earlier versions
\def\PYZat{@}
\def\PYZlb{[}
\def\PYZrb{]}
\makeatother


    % Exact colors from NB
    \definecolor{incolor}{rgb}{0.0, 0.0, 0.5}
    \definecolor{outcolor}{rgb}{0.545, 0.0, 0.0}



    
    % Prevent overflowing lines due to hard-to-break entities
    \sloppy 
    % Setup hyperref package
    \hypersetup{
      breaklinks=true,  % so long urls are correctly broken across lines
      colorlinks=true,
      urlcolor=blue,
      linkcolor=darkorange,
      citecolor=darkgreen,
      }
    % Slightly bigger margins than the latex defaults
    
    \geometry{verbose,tmargin=1in,bmargin=1in,lmargin=1in,rmargin=1in}
    
    

    \begin{document}
    
    
    \maketitle
    
    

    
    \section{BARBERO IRIARTE, PILAR - EJERCICIOS TEMA
7}\label{barbero-iriarte-pilar---ejercicios-tema-7}

    \section{Ejercicio 1.1}\label{ejercicio-1.1}

    Escribir una función cuya entrada (input) sea,

\begin{itemize}
\item
  una imagen de cualquier tipo,
\item
  unos rangos para sus píxeles, por ejemplo $23 \leq i \leq 53$ y
  $12 \leq j \leq 65$,
\end{itemize}

y cuya salida (output) sea la submatriz (tipo double) que corresponda a
los índices dados y una figura que contenga la imagen. Usar la función
para seleccionar la cabeza del cameraman en \emph{cameraman\_512.tif}.

    \begin{Verbatim}[commandchars=\\\{\}]
{\color{incolor}In [{\color{incolor}1}]:} \PY{k+kn}{from} \PY{n+nn}{\PYZus{}\PYZus{}future\PYZus{}\PYZus{}} \PY{k+kn}{import} \PY{n}{division}
        \PY{k+kn}{import} \PY{n+nn}{Image}
        \PY{k+kn}{import} \PY{n+nn}{numpy} \PY{k+kn}{as} \PY{n+nn}{np}
        \PY{k+kn}{import} \PY{n+nn}{matplotlib.pyplot} \PY{k+kn}{as} \PY{n+nn}{plt}
        \PY{o}{\PYZpc{}}\PY{k}{matplotlib} \PY{n}{inline}
        \PY{o}{\PYZpc{}}\PY{k}{pylab} \PY{n}{inline}
\end{Verbatim}

    \begin{Verbatim}[commandchars=\\\{\}]
Populating the interactive namespace from numpy and matplotlib
    \end{Verbatim}

    \begin{Verbatim}[commandchars=\\\{\}]
{\color{incolor}In [{\color{incolor}2}]:} \PY{k}{def} \PY{n+nf}{function}\PY{p}{(}\PY{n}{INPUT}\PY{p}{,} \PY{n}{down\PYZus{}i}\PY{p}{,} \PY{n}{up\PYZus{}i}\PY{p}{,} \PY{n}{down\PYZus{}j}\PY{p}{,} \PY{n}{up\PYZus{}j}\PY{p}{)}\PY{p}{:}
                \PY{n}{J}\PY{o}{=} \PY{n}{Image}\PY{o}{.}\PY{n}{open}\PY{p}{(}\PY{n}{INPUT}\PY{p}{)}
                \PY{n}{a}\PY{o}{=}\PY{n}{np}\PY{o}{.}\PY{n}{asarray}\PY{p}{(}\PY{n}{J}\PY{p}{,}\PY{n}{dtype}\PY{o}{=}\PY{n}{np}\PY{o}{.}\PY{n}{float32}\PY{p}{)}
        
                \PY{n}{cabeza} \PY{o}{=} \PY{n}{a}\PY{p}{[}\PY{n}{down\PYZus{}i}\PY{p}{:}\PY{n}{up\PYZus{}i}\PY{p}{,} \PY{n}{down\PYZus{}j}\PY{p}{:}\PY{n}{up\PYZus{}j}\PY{p}{]}
                \PY{n}{image} \PY{o}{=} \PY{n}{Image}\PY{o}{.}\PY{n}{fromarray}\PY{p}{(}\PY{n}{cabeza}\PY{o}{.}\PY{n}{astype}\PY{p}{(}\PY{n}{np}\PY{o}{.}\PY{n}{uint8}\PY{p}{)}\PY{p}{)}
        
                \PY{k}{return} \PY{p}{[}\PY{n}{a}\PY{p}{,} \PY{n}{image}\PY{p}{]}
\end{Verbatim}

    \begin{Verbatim}[commandchars=\\\{\}]
{\color{incolor}In [{\color{incolor}3}]:} \PY{n}{IMAGE}\PY{o}{=}\PY{l+s}{\PYZdq{}}\PY{l+s}{cameraman\PYZus{}512.tif}\PY{l+s}{\PYZdq{}}
\end{Verbatim}

    \begin{Verbatim}[commandchars=\\\{\}]
{\color{incolor}In [{\color{incolor}4}]:} \PY{n}{output} \PY{o}{=} \PY{n}{function}\PY{p}{(}\PY{n}{IMAGE}\PY{p}{,} \PY{l+m+mi}{70}\PY{p}{,} \PY{l+m+mi}{170}\PY{p}{,} \PY{l+m+mi}{180}\PY{p}{,} \PY{l+m+mi}{280}\PY{p}{)}
\end{Verbatim}

    \begin{Verbatim}[commandchars=\\\{\}]
{\color{incolor}In [{\color{incolor}5}]:} \PY{n}{I}\PY{o}{=}\PY{n}{Image}\PY{o}{.}\PY{n}{open}\PY{p}{(}\PY{l+s}{\PYZdq{}}\PY{l+s}{cameraman\PYZus{}512.tif}\PY{l+s}{\PYZdq{}}\PY{p}{)}
        \PY{n}{I1}\PY{o}{=}\PY{n}{I}\PY{o}{.}\PY{n}{convert}\PY{p}{(}\PY{l+s}{\PYZsq{}}\PY{l+s}{L}\PY{l+s}{\PYZsq{}}\PY{p}{)} 
        \PY{n}{a}\PY{o}{=}\PY{n}{np}\PY{o}{.}\PY{n}{asarray}\PY{p}{(}\PY{n}{I1}\PY{p}{,}\PY{n}{dtype}\PY{o}{=}\PY{n}{np}\PY{o}{.}\PY{n}{float32}\PY{p}{)}
        \PY{k}{print} \PY{n}{output}\PY{p}{[}\PY{l+m+mi}{0}\PY{p}{]} \PY{c}{\PYZsh{} matrix}
        \PY{n}{plt}\PY{o}{.}\PY{n}{figure}\PY{p}{(}\PY{n}{figsize}\PY{o}{=}\PY{p}{(}\PY{l+m+mi}{10}\PY{p}{,}\PY{l+m+mi}{5}\PY{p}{)}\PY{p}{)}
        \PY{n}{plt}\PY{o}{.}\PY{n}{subplot}\PY{p}{(}\PY{l+m+mi}{121}\PY{p}{)}
        \PY{n}{plt}\PY{o}{.}\PY{n}{imshow}\PY{p}{(}\PY{n}{a}\PY{p}{,}\PY{n}{cmap}\PY{o}{=}\PY{l+s}{\PYZsq{}}\PY{l+s}{gray}\PY{l+s}{\PYZsq{}}\PY{p}{,}\PY{n}{interpolation}\PY{o}{=}\PY{l+s}{\PYZsq{}}\PY{l+s}{nearest}\PY{l+s}{\PYZsq{}}\PY{p}{)}\PY{p}{;}
        \PY{n}{plt}\PY{o}{.}\PY{n}{subplot}\PY{p}{(}\PY{l+m+mi}{122}\PY{p}{)}
        \PY{n}{plt}\PY{o}{.}\PY{n}{imshow}\PY{p}{(}\PY{n}{output}\PY{p}{[}\PY{l+m+mi}{1}\PY{p}{]}\PY{p}{,}\PY{n}{cmap}\PY{o}{=}\PY{l+s}{\PYZsq{}}\PY{l+s}{gray}\PY{l+s}{\PYZsq{}}\PY{p}{,}\PY{n}{interpolation}\PY{o}{=}\PY{l+s}{\PYZsq{}}\PY{l+s}{nearest}\PY{l+s}{\PYZsq{}}\PY{p}{)}\PY{p}{;} \PY{c}{\PYZsh{}image}
\end{Verbatim}

    \begin{Verbatim}[commandchars=\\\{\}]
[[ 156.  157.  160. \ldots,  152.  152.  152.]
 [ 156.  157.  159. \ldots,  152.  152.  152.]
 [ 158.  157.  156. \ldots,  152.  152.  152.]
 \ldots, 
 [ 121.  123.  126. \ldots,  121.  113.  111.]
 [ 121.  123.  126. \ldots,  121.  113.  111.]
 [ 121.  123.  126. \ldots,  121.  113.  111.]]
    \end{Verbatim}

            \begin{Verbatim}[commandchars=\\\{\}]
{\color{outcolor}Out[{\color{outcolor}5}]:} <matplotlib.image.AxesImage at 0x7fac06251250>
\end{Verbatim}
        
    \begin{center}
    \adjustimage{max size={0.9\linewidth}{0.9\paperheight}}{BARBERO-tema7_files/BARBERO-tema7_7_2.png}
    \end{center}
    { \hspace*{\fill} \\}
    
    \section{Ejercicio 1.2}\label{ejercicio-1.2}

    Las máscaras son filtros geométricos de una imagen. Por ejemplo, si
queremos seleccionar una región de una imagen, podemos hacerlo
multiplicando la matriz de la imagen original por una matriz de igual
tamaño que contenga unos en la región que queremos conservar y cero en
el resto. En este ejercicio seleccionaremos una región circular de la
imagen Lena de radio $150$. Estos son los pasos:

\begin{itemize}
\itemsep1pt\parskip0pt\parsep0pt
\item
  Leer la imagen y convertirla a escala de grises y doble precisión.
\end{itemize}

    \begin{Verbatim}[commandchars=\\\{\}]
{\color{incolor}In [{\color{incolor}6}]:} \PY{n}{I}\PY{o}{=}\PY{n}{Image}\PY{o}{.}\PY{n}{open}\PY{p}{(}\PY{l+s}{\PYZdq{}}\PY{l+s}{lena.jpg}\PY{l+s}{\PYZdq{}}\PY{p}{)}
        \PY{n}{I1}\PY{o}{=}\PY{n}{I}\PY{o}{.}\PY{n}{convert}\PY{p}{(}\PY{l+s}{\PYZsq{}}\PY{l+s}{L}\PY{l+s}{\PYZsq{}}\PY{p}{)} 
        \PY{n}{a}\PY{o}{=}\PY{n}{np}\PY{o}{.}\PY{n}{asarray}\PY{p}{(}\PY{n}{I1}\PY{p}{,}\PY{n}{dtype}\PY{o}{=}\PY{n}{np}\PY{o}{.}\PY{n}{float32}\PY{p}{)}
\end{Verbatim}

    \begin{itemize}
\item
  Crear una matriz de las mismas dimensiones rellena de ceros.
\item
  Modificar esta matriz de forma que contenga unos en un círculo de
  radio $150$, es decir, si $(i-c_x)^2 + (j-c_y)^2 < 150^2$ donde
  $(c_x, c_y) = (m/2,n/2)$ es el centro de la imagen
\end{itemize}

    \begin{Verbatim}[commandchars=\\\{\}]
{\color{incolor}In [{\color{incolor}7}]:} \PY{n}{b} \PY{o}{=} \PY{n}{numpy}\PY{o}{.}\PY{n}{zeros}\PY{p}{(}\PY{n}{shape}\PY{o}{=}\PY{p}{(}\PY{l+m+mi}{512}\PY{p}{,}\PY{l+m+mi}{512}\PY{p}{)}\PY{p}{)}
        \PY{p}{[}\PY{n}{c\PYZus{}x}\PY{p}{,} \PY{n}{c\PYZus{}y}\PY{p}{]} \PY{o}{=} \PY{p}{[}\PY{l+m+mi}{512}\PY{o}{/}\PY{l+m+mi}{2}\PY{p}{,} \PY{l+m+mi}{512}\PY{o}{/}\PY{l+m+mi}{2}\PY{p}{]}
        \PY{k}{for} \PY{n}{i} \PY{o+ow}{in} \PY{n+nb}{range}\PY{p}{(}\PY{l+m+mi}{0}\PY{p}{,}\PY{l+m+mi}{512}\PY{p}{)}\PY{p}{:}
            \PY{k}{for} \PY{n}{j} \PY{o+ow}{in} \PY{n+nb}{range}\PY{p}{(}\PY{l+m+mi}{0}\PY{p}{,}\PY{l+m+mi}{512}\PY{p}{)}\PY{p}{:}
                \PY{n}{b}\PY{p}{[}\PY{n}{i}\PY{p}{]}\PY{p}{[}\PY{n}{j}\PY{p}{]} \PY{o}{=} \PY{p}{(}\PY{n}{i}\PY{o}{\PYZhy{}}\PY{n}{c\PYZus{}x}\PY{p}{)}\PY{o}{*}\PY{o}{*}\PY{l+m+mi}{2}\PY{o}{+}\PY{p}{(}\PY{n}{j}\PY{o}{\PYZhy{}}\PY{n}{c\PYZus{}y}\PY{p}{)}\PY{o}{*}\PY{o}{*}\PY{l+m+mi}{2} \PY{o}{\PYZlt{}} \PY{l+m+mi}{150}\PY{o}{*}\PY{o}{*}\PY{l+m+mi}{2}
\end{Verbatim}

    \begin{itemize}
\itemsep1pt\parskip0pt\parsep0pt
\item
  Multiplicar la imagen por la máscara (recordar que son matrices).
\end{itemize}

    \begin{Verbatim}[commandchars=\\\{\}]
{\color{incolor}In [{\color{incolor}8}]:} \PY{n}{result} \PY{o}{=} \PY{n}{a}\PY{o}{*}\PY{p}{(}\PY{n}{b}\PY{p}{)}
\end{Verbatim}

    \begin{itemize}
\itemsep1pt\parskip0pt\parsep0pt
\item
  Mostrar el resultado.
\end{itemize}

    \begin{Verbatim}[commandchars=\\\{\}]
{\color{incolor}In [{\color{incolor}9}]:} \PY{n}{plt}\PY{o}{.}\PY{n}{figure}\PY{p}{(}\PY{n}{figsize}\PY{o}{=}\PY{p}{(}\PY{l+m+mi}{10}\PY{p}{,}\PY{l+m+mi}{5}\PY{p}{)}\PY{p}{)}
        \PY{n}{plt}\PY{o}{.}\PY{n}{subplot}\PY{p}{(}\PY{l+m+mi}{121}\PY{p}{)}
        \PY{n}{plt}\PY{o}{.}\PY{n}{imshow}\PY{p}{(}\PY{n}{result}\PY{p}{,}\PY{n}{cmap}\PY{o}{=}\PY{l+s}{\PYZsq{}}\PY{l+s}{gray}\PY{l+s}{\PYZsq{}}\PY{p}{,}\PY{n}{interpolation}\PY{o}{=}\PY{l+s}{\PYZsq{}}\PY{l+s}{nearest}\PY{l+s}{\PYZsq{}}\PY{p}{)}
        \PY{n}{plt}\PY{o}{.}\PY{n}{subplot}\PY{p}{(}\PY{l+m+mi}{122}\PY{p}{)}
        \PY{n}{plt}\PY{o}{.}\PY{n}{imshow}\PY{p}{(}\PY{n}{a}\PY{p}{,}\PY{n}{cmap}\PY{o}{=}\PY{l+s}{\PYZsq{}}\PY{l+s}{gray}\PY{l+s}{\PYZsq{}}\PY{p}{,}\PY{n}{interpolation}\PY{o}{=}\PY{l+s}{\PYZsq{}}\PY{l+s}{nearest}\PY{l+s}{\PYZsq{}}\PY{p}{)}
        \PY{n}{plt}\PY{o}{.}\PY{n}{show}\PY{p}{(}\PY{p}{)}
\end{Verbatim}

    \begin{center}
    \adjustimage{max size={0.9\linewidth}{0.9\paperheight}}{BARBERO-tema7_files/BARBERO-tema7_16_0.png}
    \end{center}
    { \hspace*{\fill} \\}
    
    Cuando se multiplica por cero, se convierten a negro los píxeles de
fuera del círculo. Modifica el programa para hacer visible esos píxeles
con la mitad de su intensidad.

    \begin{Verbatim}[commandchars=\\\{\}]
{\color{incolor}In [{\color{incolor}10}]:} \PY{n}{mu}\PY{o}{=}\PY{l+m+mf}{0.5}
         \PY{n}{b\PYZus{}m} \PY{o}{=} \PY{n}{numpy}\PY{o}{.}\PY{n}{zeros}\PY{p}{(}\PY{n}{shape}\PY{o}{=}\PY{p}{(}\PY{l+m+mi}{512}\PY{p}{,}\PY{l+m+mi}{512}\PY{p}{)}\PY{p}{)}
         \PY{p}{[}\PY{n}{c\PYZus{}x}\PY{p}{,} \PY{n}{c\PYZus{}y}\PY{p}{]} \PY{o}{=} \PY{p}{[}\PY{l+m+mi}{512}\PY{o}{/}\PY{l+m+mi}{2}\PY{p}{,} \PY{l+m+mi}{512}\PY{o}{/}\PY{l+m+mi}{2}\PY{p}{]}
         \PY{k}{for} \PY{n}{i} \PY{o+ow}{in} \PY{n+nb}{range}\PY{p}{(}\PY{l+m+mi}{0}\PY{p}{,}\PY{l+m+mi}{512}\PY{p}{)}\PY{p}{:}
             \PY{k}{for} \PY{n}{j} \PY{o+ow}{in} \PY{n+nb}{range}\PY{p}{(}\PY{l+m+mi}{0}\PY{p}{,}\PY{l+m+mi}{512}\PY{p}{)}\PY{p}{:}
                 \PY{k}{if} \PY{p}{(}\PY{n}{i}\PY{o}{\PYZhy{}}\PY{n}{c\PYZus{}x}\PY{p}{)}\PY{o}{*}\PY{o}{*}\PY{l+m+mi}{2}\PY{o}{+}\PY{p}{(}\PY{n}{j}\PY{o}{\PYZhy{}}\PY{n}{c\PYZus{}y}\PY{p}{)}\PY{o}{*}\PY{o}{*}\PY{l+m+mi}{2} \PY{o}{\PYZlt{}} \PY{l+m+mi}{150}\PY{o}{*}\PY{o}{*}\PY{l+m+mi}{2}\PY{p}{:}
                     \PY{n}{b\PYZus{}m}\PY{p}{[}\PY{n}{i}\PY{p}{]}\PY{p}{[}\PY{n}{j}\PY{p}{]}\PY{o}{=}\PY{l+m+mi}{1}
                 \PY{k}{else}\PY{p}{:}
                     \PY{n}{b\PYZus{}m}\PY{p}{[}\PY{n}{i}\PY{p}{]}\PY{p}{[}\PY{n}{j}\PY{p}{]}\PY{o}{=}\PY{n}{mu}
\end{Verbatim}

    \begin{Verbatim}[commandchars=\\\{\}]
{\color{incolor}In [{\color{incolor}11}]:} \PY{n}{result\PYZus{}m} \PY{o}{=} \PY{n}{a}\PY{o}{*}\PY{p}{(}\PY{n}{b\PYZus{}m}\PY{p}{)}
\end{Verbatim}

    \begin{Verbatim}[commandchars=\\\{\}]
{\color{incolor}In [{\color{incolor}12}]:} \PY{n}{plt}\PY{o}{.}\PY{n}{figure}\PY{p}{(}\PY{n}{figsize}\PY{o}{=}\PY{p}{(}\PY{l+m+mi}{10}\PY{p}{,}\PY{l+m+mi}{5}\PY{p}{)}\PY{p}{)}
         \PY{n}{plt}\PY{o}{.}\PY{n}{subplot}\PY{p}{(}\PY{l+m+mi}{121}\PY{p}{)}
         \PY{n}{plt}\PY{o}{.}\PY{n}{imshow}\PY{p}{(}\PY{n}{result\PYZus{}m}\PY{p}{,}\PY{n}{cmap}\PY{o}{=}\PY{l+s}{\PYZsq{}}\PY{l+s}{gray}\PY{l+s}{\PYZsq{}}\PY{p}{,}\PY{n}{interpolation}\PY{o}{=}\PY{l+s}{\PYZsq{}}\PY{l+s}{nearest}\PY{l+s}{\PYZsq{}}\PY{p}{)}
         \PY{n}{plt}\PY{o}{.}\PY{n}{subplot}\PY{p}{(}\PY{l+m+mi}{122}\PY{p}{)}
         \PY{n}{plt}\PY{o}{.}\PY{n}{imshow}\PY{p}{(}\PY{n}{a}\PY{p}{,}\PY{n}{cmap}\PY{o}{=}\PY{l+s}{\PYZsq{}}\PY{l+s}{gray}\PY{l+s}{\PYZsq{}}\PY{p}{,}\PY{n}{interpolation}\PY{o}{=}\PY{l+s}{\PYZsq{}}\PY{l+s}{nearest}\PY{l+s}{\PYZsq{}}\PY{p}{)}
         \PY{n}{plt}\PY{o}{.}\PY{n}{show}\PY{p}{(}\PY{p}{)}
\end{Verbatim}

    \begin{center}
    \adjustimage{max size={0.9\linewidth}{0.9\paperheight}}{BARBERO-tema7_files/BARBERO-tema7_20_0.png}
    \end{center}
    { \hspace*{\fill} \\}
    
    \section{Ejercicio 1.3}\label{ejercicio-1.3}

    El degradado lineal es un efecto en el que se oscurece una imagen
vertical u horizontalmente. Podemos hacer esto con una máscara que sea
constante por columnas pero que tome un valor decreciente por filas,
desde uno en la primera fila hasta cero enla última. Construir dicha
matriz y aplicarla a la imagen Lena.

    \begin{Verbatim}[commandchars=\\\{\}]
{\color{incolor}In [{\color{incolor}13}]:} \PY{n}{I}\PY{o}{=}\PY{n}{Image}\PY{o}{.}\PY{n}{open}\PY{p}{(}\PY{l+s}{\PYZdq{}}\PY{l+s}{lena.jpg}\PY{l+s}{\PYZdq{}}\PY{p}{)}
         \PY{n}{I1}\PY{o}{=}\PY{n}{I}\PY{o}{.}\PY{n}{convert}\PY{p}{(}\PY{l+s}{\PYZsq{}}\PY{l+s}{L}\PY{l+s}{\PYZsq{}}\PY{p}{)} 
         \PY{n}{a}\PY{o}{=}\PY{n}{np}\PY{o}{.}\PY{n}{asarray}\PY{p}{(}\PY{n}{I1}\PY{p}{,}\PY{n}{dtype}\PY{o}{=}\PY{n}{np}\PY{o}{.}\PY{n}{float32}\PY{p}{)}
         \PY{n}{x}\PY{o}{=}\PY{n}{linspace}\PY{p}{(}\PY{l+m+mi}{0}\PY{p}{,}\PY{l+m+mi}{1}\PY{p}{,}\PY{l+m+mi}{512}\PY{p}{)}
         \PY{n}{mask} \PY{o}{=} \PY{n}{np}\PY{o}{.}\PY{n}{tile}\PY{p}{(}\PY{n}{x}\PY{p}{,} \PY{p}{(}\PY{l+m+mi}{512}\PY{p}{,} \PY{l+m+mi}{1}\PY{p}{)}\PY{p}{)}\PY{o}{.}\PY{n}{transpose}\PY{p}{(}\PY{p}{)}
         \PY{n}{result} \PY{o}{=} \PY{n}{a}\PY{o}{*}\PY{n}{mask}
\end{Verbatim}

    \begin{Verbatim}[commandchars=\\\{\}]
{\color{incolor}In [{\color{incolor}14}]:} \PY{n}{plt}\PY{o}{.}\PY{n}{figure}\PY{p}{(}\PY{n}{figsize}\PY{o}{=}\PY{p}{(}\PY{l+m+mi}{10}\PY{p}{,}\PY{l+m+mi}{5}\PY{p}{)}\PY{p}{)}
         \PY{n}{plt}\PY{o}{.}\PY{n}{subplot}\PY{p}{(}\PY{l+m+mi}{121}\PY{p}{)}
         \PY{n}{plt}\PY{o}{.}\PY{n}{imshow}\PY{p}{(}\PY{n}{result}\PY{p}{,}\PY{n}{cmap}\PY{o}{=}\PY{l+s}{\PYZsq{}}\PY{l+s}{gray}\PY{l+s}{\PYZsq{}}\PY{p}{,}\PY{n}{interpolation}\PY{o}{=}\PY{l+s}{\PYZsq{}}\PY{l+s}{nearest}\PY{l+s}{\PYZsq{}}\PY{p}{)}
         \PY{n}{plt}\PY{o}{.}\PY{n}{subplot}\PY{p}{(}\PY{l+m+mi}{122}\PY{p}{)}
         \PY{n}{plt}\PY{o}{.}\PY{n}{imshow}\PY{p}{(}\PY{n}{a}\PY{p}{,}\PY{n}{cmap}\PY{o}{=}\PY{l+s}{\PYZsq{}}\PY{l+s}{gray}\PY{l+s}{\PYZsq{}}\PY{p}{,}\PY{n}{interpolation}\PY{o}{=}\PY{l+s}{\PYZsq{}}\PY{l+s}{nearest}\PY{l+s}{\PYZsq{}}\PY{p}{)}
         \PY{n}{plt}\PY{o}{.}\PY{n}{show}\PY{p}{(}\PY{p}{)}
\end{Verbatim}

    \begin{center}
    \adjustimage{max size={0.9\linewidth}{0.9\paperheight}}{BARBERO-tema7_files/BARBERO-tema7_24_0.png}
    \end{center}
    { \hspace*{\fill} \\}
    
    \section{Ejercicio 2.1}\label{ejercicio-2.1}

    Siguiendo el esquema visto para el caso del laplaciano, escribir un
código que ejecute la minimización de la variación total. Aplicarlo a la
imagen \emph{boat\_256.png} corrompida con ruido gaussiano del $5\%$,
usando los valores de los parámetros:

$\lambda = 0.1$, $\tau = 0.75$, $\epsilon = 1 e^{-6}$, num\_iter \$
=20$, si la escala es $I\_\{255\}\$,

Mostrar el resultado en una figura.

    \textbf{Solución:}

    Aplicamos un ruido gaussiano del $5\%$ a la imagen,

    \begin{Verbatim}[commandchars=\\\{\}]
{\color{incolor}In [{\color{incolor}15}]:} \PY{n}{I} \PY{o}{=} \PY{n}{Image}\PY{o}{.}\PY{n}{open}\PY{p}{(}\PY{l+s}{\PYZdq{}}\PY{l+s}{boat\PYZus{}256.png}\PY{l+s}{\PYZdq{}}\PY{p}{)}
         \PY{n}{uc}\PY{o}{=}\PY{n}{np}\PY{o}{.}\PY{n}{asarray}\PY{p}{(}\PY{n}{I}\PY{p}{,}\PY{n}{dtype}\PY{o}{=}\PY{n}{np}\PY{o}{.}\PY{n}{float32}\PY{p}{)}
\end{Verbatim}

    \begin{Verbatim}[commandchars=\\\{\}]
{\color{incolor}In [{\color{incolor}16}]:} \PY{k}{def} \PY{n+nf}{aplicar\PYZus{}ruido}\PY{p}{(}\PY{n}{uc}\PY{p}{)}\PY{p}{:}
             \PY{n}{u0}\PY{o}{=}\PY{n}{uc}\PY{o}{+}\PY{l+m+mf}{12.}\PY{o}{*}\PY{n}{np}\PY{o}{.}\PY{n}{random}\PY{o}{.}\PY{n}{standard\PYZus{}normal}\PY{p}{(}\PY{n}{uc}\PY{o}{.}\PY{n}{shape}\PY{p}{)} \PY{c}{\PYZsh{} 0.04705882352 error}
             \PY{n}{u0}\PY{o}{=}\PY{n}{u0}\PY{o}{.}\PY{n}{clip}\PY{p}{(}\PY{n+nb}{min}\PY{o}{=}\PY{l+m+mi}{0}\PY{p}{,} \PY{n+nb}{max}\PY{o}{=}\PY{l+m+mi}{255}\PY{p}{)}
             \PY{k}{return} \PY{n}{u0}
\end{Verbatim}

    \begin{Verbatim}[commandchars=\\\{\}]
{\color{incolor}In [{\color{incolor}17}]:} \PY{n}{u0}\PY{o}{=}\PY{n}{aplicar\PYZus{}ruido}\PY{p}{(}\PY{n}{uc}\PY{p}{)}
         \PY{n}{plt}\PY{o}{.}\PY{n}{figure}\PY{p}{(}\PY{n}{figsize}\PY{o}{=}\PY{p}{(}\PY{l+m+mi}{10}\PY{p}{,}\PY{l+m+mi}{5}\PY{p}{)}\PY{p}{)}
         \PY{n}{plt}\PY{o}{.}\PY{n}{subplot}\PY{p}{(}\PY{l+m+mi}{121}\PY{p}{)}
         \PY{n}{plt}\PY{o}{.}\PY{n}{imshow}\PY{p}{(}\PY{n}{uc}\PY{p}{,} \PY{n}{cmap}\PY{o}{=}\PY{l+s}{\PYZsq{}}\PY{l+s}{gray}\PY{l+s}{\PYZsq{}}\PY{p}{,} \PY{n}{interpolation}\PY{o}{=}\PY{l+s}{\PYZsq{}}\PY{l+s}{nearest}\PY{l+s}{\PYZsq{}}\PY{p}{)}
         \PY{n}{plt}\PY{o}{.}\PY{n}{title}\PY{p}{(}\PY{l+s}{\PYZdq{}}\PY{l+s}{Boat}\PY{l+s}{\PYZdq{}}\PY{p}{)}\PY{p}{;}
         \PY{n}{plt}\PY{o}{.}\PY{n}{subplot}\PY{p}{(}\PY{l+m+mi}{122}\PY{p}{)}
         \PY{n}{plt}\PY{o}{.}\PY{n}{imshow}\PY{p}{(}\PY{n}{u0}\PY{p}{,} \PY{n}{cmap}\PY{o}{=}\PY{l+s}{\PYZsq{}}\PY{l+s}{gray}\PY{l+s}{\PYZsq{}}\PY{p}{,} \PY{n}{interpolation}\PY{o}{=}\PY{l+s}{\PYZsq{}}\PY{l+s}{nearest}\PY{l+s}{\PYZsq{}}\PY{p}{)}
         \PY{n}{plt}\PY{o}{.}\PY{n}{title}\PY{p}{(}\PY{l+s}{\PYZdq{}}\PY{l+s}{Boat con 5}\PY{l+s+si}{\PYZpc{} r}\PY{l+s}{uido}\PY{l+s}{\PYZdq{}}\PY{p}{)}\PY{p}{;}
\end{Verbatim}

    \begin{center}
    \adjustimage{max size={0.9\linewidth}{0.9\paperheight}}{BARBERO-tema7_files/BARBERO-tema7_31_0.png}
    \end{center}
    { \hspace*{\fill} \\}
    
    Definimos la función que aplica la minimización de la variación total,

    \begin{Verbatim}[commandchars=\\\{\}]
{\color{incolor}In [{\color{incolor}18}]:} \PY{k}{def} \PY{n+nf}{variacion\PYZus{}total}\PY{p}{(}\PY{n}{u0}\PY{p}{,} \PY{n}{lbda}\PY{p}{,} \PY{n}{tau}\PY{p}{,} \PY{n}{eps}\PY{p}{)}\PY{p}{:}
             \PY{n}{u}\PY{o}{=}\PY{n}{np}\PY{o}{.}\PY{n}{copy}\PY{p}{(}\PY{n}{u0}\PY{p}{)}
             \PY{c}{\PYZsh{}u=u0}
             \PY{k}{for} \PY{n}{i} \PY{o+ow}{in} \PY{n+nb}{range}\PY{p}{(}\PY{l+m+mi}{0}\PY{p}{,}\PY{l+m+mi}{20}\PY{p}{)}\PY{p}{:}
                 \PY{n}{GradU}\PY{o}{=}\PY{n}{np}\PY{o}{.}\PY{n}{gradient}\PY{p}{(}\PY{n}{u}\PY{p}{)}
                 \PY{n}{pupx}\PY{o}{=}\PY{n}{GradU}\PY{p}{[}\PY{l+m+mi}{0}\PY{p}{]}
                 \PY{n}{pupy}\PY{o}{=}\PY{n}{GradU}\PY{p}{[}\PY{l+m+mi}{1}\PY{p}{]}
                 \PY{n}{coefx} \PY{o}{=} \PY{n}{pupx} \PY{o}{/} \PY{n}{sqrt}\PY{p}{(}\PY{n}{eps}\PY{o}{*}\PY{o}{*}\PY{l+m+mi}{2} \PY{o}{+} \PY{n}{pupx}\PY{o}{*}\PY{o}{*}\PY{l+m+mi}{2} \PY{o}{+} \PY{n}{pupy}\PY{o}{*}\PY{o}{*}\PY{l+m+mi}{2}\PY{p}{)}
                 \PY{n}{coefy} \PY{o}{=} \PY{n}{pupy} \PY{o}{/} \PY{n}{sqrt}\PY{p}{(}\PY{n}{eps}\PY{o}{*}\PY{o}{*}\PY{l+m+mi}{2} \PY{o}{+} \PY{n}{pupx}\PY{o}{*}\PY{o}{*}\PY{l+m+mi}{2} \PY{o}{+} \PY{n}{pupy}\PY{o}{*}\PY{o}{*}\PY{l+m+mi}{2}\PY{p}{)}
                 \PY{n}{lap} \PY{o}{=} \PY{n}{np}\PY{o}{.}\PY{n}{gradient}\PY{p}{(}\PY{n}{coefx}\PY{p}{)}\PY{p}{[}\PY{l+m+mi}{0}\PY{p}{]} \PY{o}{+} \PY{n}{np}\PY{o}{.}\PY{n}{gradient}\PY{p}{(}\PY{n}{coefy}\PY{p}{)}\PY{p}{[}\PY{l+m+mi}{1}\PY{p}{]}        
                 \PY{n}{u} \PY{o}{=} \PY{n}{u}\PY{o}{+}\PY{n}{tau}\PY{o}{*}\PY{p}{(}\PY{n}{lap}\PY{o}{+}\PY{n}{lbda}\PY{o}{*}\PY{p}{(}\PY{n}{u0}\PY{o}{\PYZhy{}}\PY{n}{u}\PY{p}{)}\PY{p}{)}
         
             \PY{n}{u}\PY{o}{=}\PY{l+m+mi}{255}\PY{o}{*}\PY{p}{(}\PY{n}{u}\PY{o}{\PYZhy{}}\PY{n}{u}\PY{o}{.}\PY{n}{min}\PY{p}{(}\PY{p}{)}\PY{p}{)}\PY{o}{/}\PY{p}{(}\PY{n}{u}\PY{o}{.}\PY{n}{max}\PY{p}{(}\PY{p}{)}\PY{o}{\PYZhy{}}\PY{n}{u}\PY{o}{.}\PY{n}{min}\PY{p}{(}\PY{p}{)}\PY{p}{)}    
             \PY{k}{return} \PY{n}{u}
\end{Verbatim}

    \begin{Verbatim}[commandchars=\\\{\}]
{\color{incolor}In [{\color{incolor}19}]:} \PY{n}{u1}\PY{o}{=}\PY{n}{np}\PY{o}{.}\PY{n}{copy}\PY{p}{(}\PY{n}{u0}\PY{p}{)}
         \PY{n}{restaurada} \PY{o}{=} \PY{n}{variacion\PYZus{}total}\PY{p}{(}\PY{n}{u1}\PY{p}{,} \PY{l+m+mf}{0.1}\PY{p}{,} \PY{l+m+mf}{0.75}\PY{p}{,} \PY{l+m+mi}{10}\PY{o}{*}\PY{o}{*}\PY{p}{(}\PY{o}{\PYZhy{}}\PY{l+m+mi}{6}\PY{p}{)}\PY{p}{)}
\end{Verbatim}

    \begin{Verbatim}[commandchars=\\\{\}]
{\color{incolor}In [{\color{incolor}20}]:} \PY{n}{plt}\PY{o}{.}\PY{n}{figure}\PY{p}{(}\PY{l+m+mi}{1}\PY{p}{,} \PY{n}{figsize}\PY{o}{=}\PY{p}{(}\PY{l+m+mi}{20}\PY{p}{,}\PY{l+m+mi}{10}\PY{p}{)}\PY{p}{)}
         \PY{n}{plt}\PY{o}{.}\PY{n}{subplot}\PY{p}{(}\PY{l+m+mi}{131}\PY{p}{)}
         \PY{n}{plt}\PY{o}{.}\PY{n}{imshow}\PY{p}{(}\PY{n}{uc}\PY{p}{,}\PY{n}{cmap}\PY{o}{=}\PY{l+s}{\PYZsq{}}\PY{l+s}{gray}\PY{l+s}{\PYZsq{}}\PY{p}{,}\PY{n}{interpolation}\PY{o}{=}\PY{l+s}{\PYZsq{}}\PY{l+s}{nearest}\PY{l+s}{\PYZsq{}}\PY{p}{)}
         \PY{n}{plt}\PY{o}{.}\PY{n}{title}\PY{p}{(}\PY{l+s}{\PYZsq{}}\PY{l+s}{Imagen original}\PY{l+s}{\PYZsq{}}\PY{p}{)}\PY{p}{;}
         \PY{n}{plt}\PY{o}{.}\PY{n}{subplot}\PY{p}{(}\PY{l+m+mi}{132}\PY{p}{)}
         \PY{n}{plt}\PY{o}{.}\PY{n}{imshow}\PY{p}{(}\PY{n}{u0}\PY{p}{,}\PY{n}{cmap}\PY{o}{=}\PY{l+s}{\PYZsq{}}\PY{l+s}{gray}\PY{l+s}{\PYZsq{}}\PY{p}{,}\PY{n}{interpolation}\PY{o}{=}\PY{l+s}{\PYZsq{}}\PY{l+s}{nearest}\PY{l+s}{\PYZsq{}}\PY{p}{)}
         \PY{n}{plt}\PY{o}{.}\PY{n}{title}\PY{p}{(}\PY{l+s}{\PYZsq{}}\PY{l+s}{Imagen con ruido}\PY{l+s}{\PYZsq{}}\PY{p}{)}\PY{p}{;}
         \PY{n}{plt}\PY{o}{.}\PY{n}{subplot}\PY{p}{(}\PY{l+m+mi}{133}\PY{p}{)}
         \PY{n}{plt}\PY{o}{.}\PY{n}{imshow}\PY{p}{(}\PY{n}{restaurada}\PY{p}{,}\PY{n}{cmap}\PY{o}{=}\PY{l+s}{\PYZsq{}}\PY{l+s}{gray}\PY{l+s}{\PYZsq{}}\PY{p}{,}\PY{n}{interpolation}\PY{o}{=}\PY{l+s}{\PYZsq{}}\PY{l+s}{nearest}\PY{l+s}{\PYZsq{}}\PY{p}{)}
         \PY{n}{plt}\PY{o}{.}\PY{n}{title}\PY{p}{(}\PY{l+s}{\PYZsq{}}\PY{l+s}{Restaurada con valores 0.1, 0.75, 1.e\PYZhy{}6}\PY{l+s}{\PYZsq{}}\PY{p}{)}\PY{p}{;}
\end{Verbatim}

    \begin{center}
    \adjustimage{max size={0.9\linewidth}{0.9\paperheight}}{BARBERO-tema7_files/BARBERO-tema7_35_0.png}
    \end{center}
    { \hspace*{\fill} \\}
    
    \section{Ejercicio 2.2}\label{ejercicio-2.2}

    Calcular la derivada bidireccional del funcional de variación total,

\[ J(v) = \int_{\Omega} |\nabla v(x) | \text{d}x\]

\emph{Indicación:} Usar que $|s| = \sqrt{|s|^2}$.

    \textbf{Respuesta:}

    Una primera manera, aplicando la definición de derivada direccinal y
aplicando $|a| = \frac{|a|^2}{|a|}$,

\[
< J'(v), \varphi > = \lim\limits_{h\rightarrow 0} \frac{J(v+h\varphi) - J(v)}{h} 
\]

\[
= \lim\limits_{h\rightarrow 0} \frac{1}{h} \Big[\int_{\Omega} |\nabla(v(x) + h\varphi(x))|\text{d}x - \int_{\Omega} |\nabla v(x)|\text{d}x \Big] 
\]

\[
= \lim\limits_{h\rightarrow 0} \int_{\Omega}\frac{1}{h} (|\nabla v(x) + h\nabla \varphi(x)| - |\nabla v(x)|) \text{d}x
\]

\[
= \lim\limits_{h\rightarrow 0} \int_{\Omega}\frac{1}{h} (\frac{|\nabla v(x) + h\nabla \varphi(x)|^2}{|\nabla v(x) + h\nabla \varphi(x)|} - \frac{|\nabla v(x)|^2}{|\nabla v(x)|} ) \text{d}x
\]

\[
= \lim\limits_{h\rightarrow 0} \int_{\Omega}\frac{1}{h} \frac{|\nabla v| |\nabla v + h\nabla\varphi|^2 - |\nabla v + h \nabla \varphi| |\nabla v|^2}{|\nabla v| |\nabla v + h \nabla \varphi|} \text{d}x 
\]

\[
= \lim\limits_{h\rightarrow 0} \int_{\Omega}\frac{1}{h} \frac{|\nabla v| (|\nabla v|^2 + 2h\nabla v\nabla \varphi + h^2 |\nabla \varphi|^2) - |\nabla v|^3 - h |\nabla v|^2|\nabla \varphi|}{|\nabla v|^2 + h |\nabla v| |\nabla \varphi|} \text{d}x \]

\[
= \lim\limits_{h\rightarrow 0} \int_{\Omega}\frac{1}{h} \frac{h \nabla v |\nabla v| \nabla \varphi + h^2 |\nabla v| |\nabla \varphi|^2}{|\nabla v|^2 + h |\nabla v| |\nabla \varphi|} \text{d}x
\]

\[
= \int_{\Omega} |\nabla \varphi| \frac{\nabla v}{|\nabla v|}\text{d}x
\]

    Otra forma es multiplicando y diviendo por el conjugado:

    Tratamos las equivalencias,

\[
|\nabla v + h\nabla \varphi| - |\nabla v| \sim \frac{|\nabla v + h\nabla \varphi|^2 - |\nabla v|^2 }{|\nabla v + h\nabla \varphi| + |\nabla v|}
\sim \frac{|\nabla v|^2 + 2 \nabla v \nabla \varphi + h^2|\nabla \varphi|^2 - |\nabla v|^2}{|\nabla v + h\nabla \varphi| + |\nabla v|} 
\]

\[
\sim \frac{2 \nabla v \cdot \nabla \varphi + h^2 |\nabla \varphi|^2 }{|\nabla v + h\nabla \varphi| + |\nabla v|} 
\sim h |\nabla \varphi| \frac{2\nabla v + h|\nabla \varphi|}{|\nabla v + h\nabla \varphi| + |\nabla v|} \xrightarrow{h\rightarrow 0} \nabla \varphi \frac{\nabla v}{|\nabla v|}
\]

    Concluyendo,

\[ < J'(v), \varphi > = \int_{\Omega} \nabla \varphi \frac{\nabla v}{|\nabla v|}\]

    \section{Ejercicio 3}\label{ejercicio-3}

    En el \textbf{Ejercicio 2.1} hemos realizado un código para la
eliminación de ruido mediante la minimización de la variación total. En
este, comenzaremos por escribir un código que implemente el filtro de
Yaroslavsky. Una vez compuestos los dos códigos, vamos a comparar los
resultados aplicados a la imagen \emph{boat\_256.png}, siguiendo lo
visto en el Laboratorio 3. Haremos lo siguiente:

\begin{itemize}
\itemsep1pt\parskip0pt\parsep0pt
\item
  Leer la imagen \emph{boat\_256.png} y transformarla a doble precisión,
  escala de grises.
\end{itemize}

    \begin{Verbatim}[commandchars=\\\{\}]
{\color{incolor}In [{\color{incolor}21}]:} \PY{n}{I}\PY{o}{=}\PY{n}{Image}\PY{o}{.}\PY{n}{open}\PY{p}{(}\PY{l+s}{\PYZdq{}}\PY{l+s}{boat\PYZus{}256.png}\PY{l+s}{\PYZdq{}}\PY{p}{)}
         \PY{n}{I1}\PY{o}{=}\PY{n}{I}\PY{o}{.}\PY{n}{convert}\PY{p}{(}\PY{l+s}{\PYZsq{}}\PY{l+s}{L}\PY{l+s}{\PYZsq{}}\PY{p}{)} 
         \PY{n}{a}\PY{o}{=}\PY{n}{np}\PY{o}{.}\PY{n}{asarray}\PY{p}{(}\PY{n}{I1}\PY{p}{,}\PY{n}{dtype}\PY{o}{=}\PY{n}{np}\PY{o}{.}\PY{n}{float32}\PY{p}{)}
\end{Verbatim}

    \begin{itemize}
\itemsep1pt\parskip0pt\parsep0pt
\item
  Crear una nueva imagen añadiendo a la anterior un ruido gaussiano del
  $5\%$.
\end{itemize}

    \begin{Verbatim}[commandchars=\\\{\}]
{\color{incolor}In [{\color{incolor}22}]:} \PY{n}{a0}\PY{o}{=}\PY{n}{aplicar\PYZus{}ruido}\PY{p}{(}\PY{n}{a}\PY{p}{)}
         \PY{n}{a1}\PY{o}{=}\PY{n}{np}\PY{o}{.}\PY{n}{copy}\PY{p}{(}\PY{n}{a0}\PY{p}{)}
         \PY{n}{a2}\PY{o}{=}\PY{n}{np}\PY{o}{.}\PY{n}{copy}\PY{p}{(}\PY{n}{a0}\PY{p}{)}
         \PY{n}{plt}\PY{o}{.}\PY{n}{figure}\PY{p}{(}\PY{l+m+mi}{1}\PY{p}{,} \PY{n}{figsize}\PY{o}{=}\PY{p}{(}\PY{l+m+mi}{10}\PY{p}{,}\PY{l+m+mi}{6}\PY{p}{)}\PY{p}{)}
         \PY{n}{plt}\PY{o}{.}\PY{n}{subplot}\PY{p}{(}\PY{l+m+mi}{121}\PY{p}{)}
         \PY{n}{plt}\PY{o}{.}\PY{n}{imshow}\PY{p}{(}\PY{n}{a}\PY{p}{,}\PY{n}{cmap}\PY{o}{=}\PY{l+s}{\PYZsq{}}\PY{l+s}{gray}\PY{l+s}{\PYZsq{}}\PY{p}{,}\PY{n}{interpolation}\PY{o}{=}\PY{l+s}{\PYZsq{}}\PY{l+s}{nearest}\PY{l+s}{\PYZsq{}}\PY{p}{)}
         \PY{n}{plt}\PY{o}{.}\PY{n}{title}\PY{p}{(}\PY{l+s}{\PYZsq{}}\PY{l+s}{Imagen original}\PY{l+s}{\PYZsq{}}\PY{p}{)}\PY{p}{;}
         \PY{n}{plt}\PY{o}{.}\PY{n}{subplot}\PY{p}{(}\PY{l+m+mi}{122}\PY{p}{)}
         \PY{n}{plt}\PY{o}{.}\PY{n}{imshow}\PY{p}{(}\PY{n}{a0}\PY{p}{,}\PY{n}{cmap}\PY{o}{=}\PY{l+s}{\PYZsq{}}\PY{l+s}{gray}\PY{l+s}{\PYZsq{}}\PY{p}{,}\PY{n}{interpolation}\PY{o}{=}\PY{l+s}{\PYZsq{}}\PY{l+s}{nearest}\PY{l+s}{\PYZsq{}}\PY{p}{)}
         \PY{n}{plt}\PY{o}{.}\PY{n}{title}\PY{p}{(}\PY{l+s}{\PYZsq{}}\PY{l+s}{Imagen con ruido}\PY{l+s}{\PYZsq{}}\PY{p}{)}\PY{p}{;}
\end{Verbatim}

    \begin{center}
    \adjustimage{max size={0.9\linewidth}{0.9\paperheight}}{BARBERO-tema7_files/BARBERO-tema7_47_0.png}
    \end{center}
    { \hspace*{\fill} \\}
    
    \begin{itemize}
\itemsep1pt\parskip0pt\parsep0pt
\item
  Usar el código de variación total con los parámetros
\end{itemize}

$\lambda = 0.1$, $\tau = 0.75$, $\epsilon = 1 e^{-6}$, num\_iter\$
=20$, si la escala es $I\_\{255\}\$,

    \begin{Verbatim}[commandchars=\\\{\}]
{\color{incolor}In [{\color{incolor}23}]:} \PY{n}{a\PYZus{}v} \PY{o}{=} \PY{n}{variacion\PYZus{}total}\PY{p}{(}\PY{n}{a1}\PY{p}{,} \PY{l+m+mf}{0.1}\PY{p}{,} \PY{l+m+mf}{0.75}\PY{p}{,} \PY{l+m+mi}{10}\PY{o}{*}\PY{o}{*}\PY{p}{(}\PY{o}{\PYZhy{}}\PY{l+m+mi}{6}\PY{p}{)}\PY{p}{)}
\end{Verbatim}

    \begin{itemize}
\itemsep1pt\parskip0pt\parsep0pt
\item
  Usar el filtro Yaroslavsky con los parámetros $h = \rho = 30$, si la
  escala es $I_{255}$
\end{itemize}

    \emph{Indicación:} Para la integración del filtro de Yaroslavsky
utilizamos la aproximación siguiente,

\[ \int_{B(x,\rho)} f(x)\text{d}x = \sum_{x\in B(x,\rho)} f(x) \]

\emph{Nota:} El filtro de Yaroslavsky es algo más sencillo de programar
que el bilateral, dado que la integración se limita directamente al
cuadrado $B(x,\rho)$. En el filtro bilateral, en principio, la
integración ha de realizarse sobre toda la imagen. Sin embargo, puesto
que la función gaussiana decae a cero muy rápidamente, normalmente se
limita el recinto de integración a un cuadrado de radio $2\rho$.

    \begin{Verbatim}[commandchars=\\\{\}]
{\color{incolor}In [{\color{incolor}24}]:} \PY{k}{def} \PY{n+nf}{yaroslavsky}\PY{p}{(}\PY{n}{b0}\PY{p}{,} \PY{n}{h}\PY{p}{,} \PY{n}{rho}\PY{p}{)}\PY{p}{:}
             \PY{n}{b}\PY{o}{=}\PY{n}{np}\PY{o}{.}\PY{n}{copy}\PY{p}{(}\PY{n}{b0}\PY{p}{)}
             \PY{k}{for} \PY{n}{i} \PY{o+ow}{in} \PY{n+nb}{range}\PY{p}{(}\PY{l+m+mi}{0}\PY{p}{,}\PY{l+m+mi}{256}\PY{p}{)}\PY{p}{:}
                 \PY{k}{for} \PY{n}{j} \PY{o+ow}{in} \PY{n+nb}{range}\PY{p}{(}\PY{l+m+mi}{0}\PY{p}{,}\PY{l+m+mi}{256}\PY{p}{)}\PY{p}{:}
                     \PY{n}{iMin}\PY{o}{=}\PY{n+nb}{max}\PY{p}{(}\PY{n}{i} \PY{o}{\PYZhy{}} \PY{n}{rho}\PY{p}{,} \PY{l+m+mi}{1}\PY{p}{)}
                     \PY{n}{iMax}\PY{o}{=}\PY{n+nb}{min}\PY{p}{(}\PY{n}{i} \PY{o}{+} \PY{n}{rho}\PY{p}{,} \PY{l+m+mi}{256}\PY{p}{)}
                     \PY{n}{jMin}\PY{o}{=}\PY{n+nb}{max}\PY{p}{(}\PY{n}{j} \PY{o}{\PYZhy{}} \PY{n}{rho}\PY{p}{,} \PY{l+m+mi}{1}\PY{p}{)}
                     \PY{n}{jMax}\PY{o}{=}\PY{n+nb}{min}\PY{p}{(}\PY{n}{j} \PY{o}{+} \PY{n}{rho}\PY{p}{,} \PY{l+m+mi}{256}\PY{p}{)}
                     \PY{n}{I}\PY{o}{=}\PY{n}{b0}\PY{p}{[}\PY{n}{iMin}\PY{p}{:}\PY{n}{iMax}\PY{p}{,}\PY{n}{jMin}\PY{p}{:}\PY{n}{jMax}\PY{p}{]}
                     \PY{n}{H}\PY{o}{=}\PY{n}{exp}\PY{p}{(}\PY{o}{\PYZhy{}}\PY{p}{(}\PY{n}{I}\PY{o}{\PYZhy{}}\PY{n}{b0}\PY{p}{[}\PY{n}{i}\PY{p}{]}\PY{p}{[}\PY{n}{j}\PY{p}{]}\PY{p}{)}\PY{o}{*}\PY{o}{*}\PY{l+m+mi}{2}\PY{o}{/}\PY{n}{h}\PY{o}{*}\PY{o}{*}\PY{l+m+mi}{2}\PY{p}{)}
                     \PY{n}{b}\PY{p}{[}\PY{n}{i}\PY{p}{]}\PY{p}{[}\PY{n}{j}\PY{p}{]}\PY{o}{=}\PY{n+nb}{sum}\PY{p}{(}\PY{n}{H}\PY{o}{*}\PY{n}{I}\PY{p}{)}\PY{o}{/}\PY{n+nb}{sum}\PY{p}{(}\PY{n}{H}\PY{p}{)}
             \PY{k}{return} \PY{n}{b}
\end{Verbatim}

    \begin{Verbatim}[commandchars=\\\{\}]
{\color{incolor}In [{\color{incolor}25}]:} \PY{n}{a\PYZus{}y} \PY{o}{=} \PY{n}{yaroslavsky}\PY{p}{(}\PY{n}{a2}\PY{p}{,} \PY{l+m+mi}{30}\PY{p}{,} \PY{l+m+mi}{30}\PY{p}{)}
\end{Verbatim}

    De este modo, el cuadrado en el que se realiza la integración es de
tamaño $61×61$.

    \begin{itemize}
\itemsep1pt\parskip0pt\parsep0pt
\item
  Visualizar las imágenes resultantes y sus curvas de nivel.
\end{itemize}

    \begin{Verbatim}[commandchars=\\\{\}]
{\color{incolor}In [{\color{incolor}31}]:} \PY{n}{plt}\PY{o}{.}\PY{n}{figure}\PY{p}{(}\PY{l+m+mi}{1}\PY{p}{,} \PY{n}{figsize}\PY{o}{=}\PY{p}{(}\PY{l+m+mi}{15}\PY{p}{,}\PY{l+m+mi}{10}\PY{p}{)}\PY{p}{)}
         \PY{n}{plt}\PY{o}{.}\PY{n}{subplot}\PY{p}{(}\PY{l+m+mi}{221}\PY{p}{)}
         \PY{n}{plt}\PY{o}{.}\PY{n}{imshow}\PY{p}{(}\PY{n}{a\PYZus{}v}\PY{p}{,}\PY{n}{cmap}\PY{o}{=}\PY{l+s}{\PYZsq{}}\PY{l+s}{gray}\PY{l+s}{\PYZsq{}}\PY{p}{,}\PY{n}{interpolation}\PY{o}{=}\PY{l+s}{\PYZsq{}}\PY{l+s}{nearest}\PY{l+s}{\PYZsq{}}\PY{p}{)}
         \PY{n}{plt}\PY{o}{.}\PY{n}{title}\PY{p}{(}\PY{l+s}{\PYZsq{}}\PY{l+s}{Restaurada variacion total: 0.1, 0.75, 1.e\PYZhy{}6}\PY{l+s}{\PYZsq{}}\PY{p}{)}\PY{p}{;}
         \PY{n}{plt}\PY{o}{.}\PY{n}{subplot}\PY{p}{(}\PY{l+m+mi}{222}\PY{p}{)}
         \PY{n}{plt}\PY{o}{.}\PY{n}{imshow}\PY{p}{(}\PY{n}{a\PYZus{}y}\PY{p}{,}\PY{n}{cmap}\PY{o}{=}\PY{l+s}{\PYZsq{}}\PY{l+s}{gray}\PY{l+s}{\PYZsq{}}\PY{p}{,}\PY{n}{interpolation}\PY{o}{=}\PY{l+s}{\PYZsq{}}\PY{l+s}{nearest}\PY{l+s}{\PYZsq{}}\PY{p}{)}
         \PY{n}{plt}\PY{o}{.}\PY{n}{title}\PY{p}{(}\PY{l+s}{\PYZsq{}}\PY{l+s}{Restaurada Yaroslavsky 30, 30}\PY{l+s}{\PYZsq{}}\PY{p}{)}\PY{p}{;}
         \PY{n}{plt}\PY{o}{.}\PY{n}{subplot}\PY{p}{(}\PY{l+m+mi}{223}\PY{p}{)}
         \PY{n}{plt}\PY{o}{.}\PY{n}{contour}\PY{p}{(}\PY{n}{a\PYZus{}v}\PY{p}{,}\PY{n}{origin}\PY{o}{=}\PY{l+s}{\PYZsq{}}\PY{l+s}{upper}\PY{l+s}{\PYZsq{}}\PY{p}{)}
         \PY{n}{plt}\PY{o}{.}\PY{n}{axis}\PY{p}{(}\PY{l+s}{\PYZsq{}}\PY{l+s}{image}\PY{l+s}{\PYZsq{}}\PY{p}{)}\PY{p}{,} \PY{n}{plt}\PY{o}{.}\PY{n}{axis}\PY{p}{(}\PY{l+s}{\PYZsq{}}\PY{l+s}{off}\PY{l+s}{\PYZsq{}}\PY{p}{)}
         \PY{n}{plt}\PY{o}{.}\PY{n}{title}\PY{p}{(}\PY{l+s}{\PYZsq{}}\PY{l+s}{Curva de nivel: restaurada con VT}\PY{l+s}{\PYZsq{}}\PY{p}{)}\PY{p}{;}
         \PY{n}{plt}\PY{o}{.}\PY{n}{subplot}\PY{p}{(}\PY{l+m+mi}{224}\PY{p}{)}
         \PY{n}{plt}\PY{o}{.}\PY{n}{contour}\PY{p}{(}\PY{n}{a\PYZus{}y}\PY{p}{,}\PY{n}{origin}\PY{o}{=}\PY{l+s}{\PYZsq{}}\PY{l+s}{upper}\PY{l+s}{\PYZsq{}}\PY{p}{)}
         \PY{n}{plt}\PY{o}{.}\PY{n}{axis}\PY{p}{(}\PY{l+s}{\PYZsq{}}\PY{l+s}{image}\PY{l+s}{\PYZsq{}}\PY{p}{)}\PY{p}{,} \PY{n}{plt}\PY{o}{.}\PY{n}{axis}\PY{p}{(}\PY{l+s}{\PYZsq{}}\PY{l+s}{off}\PY{l+s}{\PYZsq{}}\PY{p}{)}
         \PY{n}{plt}\PY{o}{.}\PY{n}{title}\PY{p}{(}\PY{l+s}{\PYZsq{}}\PY{l+s}{Curva de nivel: restaurada con Yaroslavsky}\PY{l+s}{\PYZsq{}}\PY{p}{)}\PY{p}{;}
\end{Verbatim}

    \begin{center}
    \adjustimage{max size={0.9\linewidth}{0.9\paperheight}}{BARBERO-tema7_files/BARBERO-tema7_56_0.png}
    \end{center}
    { \hspace*{\fill} \\}
    
    Calcular los PSNR's de las imágenes restauradas respecto la imagen
original sin ruido. Obsérvese que PSNR no es una función predefinida,
hay que escribirla.

    \begin{Verbatim}[commandchars=\\\{\}]
{\color{incolor}In [{\color{incolor}27}]:} \PY{k}{def} \PY{n+nf}{PSNR}\PY{p}{(}\PY{n}{u0}\PY{p}{,} \PY{n}{u}\PY{p}{)}\PY{p}{:}
             \PY{n}{MSE} \PY{o}{=} \PY{p}{(}\PY{p}{(}\PY{n}{u0} \PY{o}{\PYZhy{}} \PY{n}{u}\PY{p}{)} \PY{o}{*}\PY{o}{*} \PY{l+m+mi}{2}\PY{p}{)}\PY{o}{.}\PY{n}{mean}\PY{p}{(}\PY{n}{axis}\PY{o}{=}\PY{n+nb+bp}{None}\PY{p}{)}
             \PY{k}{return} \PY{l+m+mi}{20}\PY{o}{*}\PY{n}{log10}\PY{p}{(}\PY{n}{u}\PY{o}{.}\PY{n}{max}\PY{p}{(}\PY{p}{)}\PY{o}{/}\PY{n}{sqrt}\PY{p}{(}\PY{n}{MSE}\PY{p}{)}\PY{p}{)}
\end{Verbatim}

    \begin{itemize}
\itemsep1pt\parskip0pt\parsep0pt
\item
  PSNR de la imagen restaurada por variación total,
\end{itemize}

    \begin{Verbatim}[commandchars=\\\{\}]
{\color{incolor}In [{\color{incolor}28}]:} \PY{k}{print} \PY{n}{PSNR}\PY{p}{(}\PY{n}{a}\PY{p}{,} \PY{n}{a\PYZus{}v}\PY{p}{)}
\end{Verbatim}

    \begin{Verbatim}[commandchars=\\\{\}]
27.3990420599
    \end{Verbatim}

    \begin{itemize}
\itemsep1pt\parskip0pt\parsep0pt
\item
  PSNR de la imagen restaurada por yaroslavsky,
\end{itemize}

    \begin{Verbatim}[commandchars=\\\{\}]
{\color{incolor}In [{\color{incolor}29}]:} \PY{k}{print} \PY{n}{PSNR}\PY{p}{(}\PY{n}{a}\PY{p}{,} \PY{n}{a\PYZus{}y}\PY{p}{)} 
\end{Verbatim}

    \begin{Verbatim}[commandchars=\\\{\}]
26.8804136806
    \end{Verbatim}


    % Add a bibliography block to the postdoc
    
    
    
    \end{document}
