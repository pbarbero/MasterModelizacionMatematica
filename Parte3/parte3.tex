\documentclass[a4paper,12pt]{article}
\usepackage{amsmath, amssymb}

\usepackage[utf8]{inputenc}
%\usepackage[spanish]{babel}
\usepackage{amsmath}
\usepackage{amsfonts}
\usepackage{amssymb}
\usepackage{graphicx}
\usepackage{exercise}

\author{Pilar Barbero Iriarte}

\newenvironment{exercise}[1]% environment name
{% begin code
  \par\vspace{\baselineskip}\noindent
  \textbf{Ejercicio (#1)}\begin{itshape}%
  \par\vspace{\baselineskip}\noindent\ignorespaces
}%
{% end code
  \end{itshape}\ignorespacesafterend
}


\begin{document}

\begin{titlepage}
\begin{center}


% Upper part of the page. The '~' is needed because \\
% only works if a paragraph has started.

\textsc{\LARGE M\'aster en Modelizaci\'on \\e Investigaci\'on Matem\'atica,\\ Estad\'istica y Computaci\'on }\\[1.5cm]
{\large \today}

\textsc{Ejercicios Tercera Parte}\\[0.5cm]

% Title
\vfill

{ \huge \bfseries Modelos de Log\'istica \\[0.4cm] }

\vfill


\includegraphics[width=0.5\textwidth]{../logoUZ.png}~\\[1cm]

% Author and supervisor
\noindent
\begin{minipage}{0.4\textwidth}
\begin{flushleft} \large
\emph{Autor:}\\
Pilar Barbero Iriarte 
\end{flushleft}
\end{minipage}%
\begin{minipage}{0.4\textwidth}
\begin{flushright} \large
\emph{Profesor:} \\
Javier L\'opez Lorente
\end{flushright}
\end{minipage}

% Bottom of the page
\end{center}


\end{titlepage}

\pagebreak
\tableofcontents
\pagebreak

\section{Ejercicio}
\begin{exercise}{1}
Tenemos que gestionar el aprovisionamiento de un producto para el que tenemos un sistema de revisión continuo. La demanda del producto es continua (es decir, no llegan grandes pedidos de una vez) y estacionaria a lo largo del tiempo; disponemos de datos de las últimas 40 semanas que se adjuntan en el fichero demanda1.xls. El lead-time no es fijo, pero disponemos de datos de los tiempos de entrega, en días, de los últimos 30 pedidos (se adjuntan en el fichero). Los costes del proceso son: coste por hacer un pedido 400 euros; coste por unidad 10 euros, coste por almacenamiento 1.5 euros por unidad y semana.\\

En caso de ruptura del stock, las unidades retrasadas se dejan sin entregar y se estima un coste de 1000 euros cada vez que se rompe el stock. Se trabaja 6 días a la semana todas las semanas del año y en el lead-time no se tienen en cuenta los domingos (es decir, el lunes es el primer día después del sábado).\\

\begin{itemize}

\item[A.] Se quiere que no más del 2\% de las unidades se entreguen con retraso. Calcular la política de aprovisionamiento óptima (es decir, de menor costo) sujeta a la restricción de unidades no entregadas. ¿Cuál es el valor del stock de seguridad? ¿Cuál es la probabilidad de ruptura del stock en cada ciclo? Determina los costes anuales de almacenamiento, de pedido y de material. 

\item[B.] Si quitamos la restricción del apartado (A) de que no más del 2\% de las unidades se entreguen con retraso, calcula la política de aprovisionamiento óptima (es decir, de menor costo) ¿Cuál es el valor del stock de seguridad? ¿Cuál es la probabilidad de ruptura del stock en cada ciclo? Determina los costes anuales de almacenamiento, de pedido y de material.

\item[C.] Supongamos ahora que el precio de las unidades tiene un descuento, de manera que las primeras 300 unidades se pagan a 10 euros, las siguientes hasta 500 se pagan a 9 euros y las siguientes a partir de 500 a 8 euros. Responde a las preguntas del apartado (A) en esta nueva situación.
\end{itemize}

\end{exercise}
\end{document}