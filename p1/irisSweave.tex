\documentclass[fleqn, letter, 10pt]{article}
%\usepackage[round,longnamesfirst]{natbib}
\usepackage[left=2cm,top=2cm,right=2cm,nohead,nofoot]{geometry} 
\usepackage{graphicx,keyval,thumbpdf,url}
\usepackage{hyperref}
\usepackage{Sweave}

\AtBeginDocument{\setkeys{Gin}{width=0.6\textwidth}}

\setlength{\parindent}{0mm}
\setlength{\parskip}{3mm plus2mm minus2mm}

\usepackage[utf8]{inputenc}

%% end of declarations %%%%%%%%%%%%%%%%%%%%%%%%%%%%%%%%%%%%%%%%%%%%%%%
\usepackage{amsmath}
\usepackage{amsfonts}


\newcommand{\strong}[1]{{\normalfont\fontseries{b}\selectfont #1}}
\newcommand{\class}[1]{\mbox{\textsf{#1}}}
\newcommand{\func}[1]{\mbox{\texttt{#1()}}}
\newcommand{\code}[1]{\mbox{\texttt{#1}}}
\newcommand{\pkg}[1]{\strong{#1}}
\newcommand{\samp}[1]{`\mbox{\texttt{#1}}'}
\newcommand{\proglang}[1]{\textsf{#1}}
\newcommand{\set}[1]{\mathcal{#1}}
\newcommand{\sQuote}[1]{`{#1}'}
\newcommand{\dQuote}[1]{``{#1}''}
\newcommand\R{{\mathbb{R}}}

\DeclareMathOperator*{\argmin}{argmin}
\DeclareMathOperator*{\argmax}{argmax}



\begin{document}
\Sconcordance{concordance:irisSweave.tex:irisSweave.Rnw:%
1 50 1 1 6 23 1 1 2 1 0 1 1 5 0 1 2 5 0 1 2 5 0 1 1 5 0 1 1 5 %
0 1 2 5 0 1 1 6 0 1 2 1 1 5 0 1 2 1 1 6 0 1 2 4 0 1 2 3 1 1 2 %
1 0 1 1 5 0 1 2 1 1 1 2 6 0 1 1 6 0 1 2 10 0 1 1 10 0 1 1 10 0 %
1 1 10 0 1 2 6 0 1 2 6 0 1 2 1 1 1 2 5 0 1 2 1 1 1 2 5 0 1 2 1 %
1 1 3 1 0 1 1 17 0 1 1 5 0 1 1 5 0 1 1 5 0 1 2 8 0 1 2 17 0 1 %
2 13 0 1 2 1 1 1 2 1 0 1 1 8 0 1 2 7 0 1 2 4 1 1 2 11 0 1 1 5 %
0 1 2 1 1 10 0 1 1 5 0 1 2 6 0 1 1 6 0 1 1 7 0 1 2 5 1 1 2 1 0 %
1 1 1 2 11 0 1 2 3 1 1 2 1 0 1 2 2 1 7 0 1 2 1 1 1 2 1 0 2 1 7 %
0 1 2 4 1 1 2 1 0 2 1 6 0 1 2 1 1 1 2 1 0 2 1 6 0 1 2 3 1 1 2 %
6 0 1 1 6 0 2 2 1 0 1 1 6 0 1 2 3 1 1 2 4 0 1 2 1 1 1 2 1 0 1 %
5 11 0 1 2 2 1 1 2 8 0 1 2 4 1 1 4 3 0 1 2 7 0 1 2 6 0 1 2 2 1 %
1 2 1 0 1 2 1 1 6 0 1 2 13 1 1 2 1 0 1 1 5 0 1 1 5 0 1 2 1 1 7 %
0 1 2 6 0 1 3 1 1 6 0 1 2 18 1 1 3 1 0 1 1 11 0 1 2 15 0 1 2 1 %
1 1 2 4 0 1 2 1 1 1 2 1 0 1 1 4 0 1 2 2 1 1 2 5 0 1 2 1 1 1 2 %
1 0 1 1 4 0 1 2 5 1 1 2 1 0 2 1 4 0 1 3 3 1 1 2 1 0 1 1 10 0 1 %
2 4 0 1 2 2 1 1 2 1 0 2 1 4 0 1 2 4 1 1 2 1 0 1 1 28 0 1 2 9 0 %
1 2 2 1 1 2 1 0 1 1 13 0 1 1 9 0 1 2 4 0 1 2 2 1 1 2 1 0 1 1 %
10 0 1 2 1 1 10 0 1 2 4 0 2 2 1 0 1 1 1 2 10 0 1 2 6 1 1 2 1 0 %
4 1 1 3 2 1 13 0 1 1 6 0 1 2 3 1 1 2 1 0 1 2 1 1 9 0 1 2 6 0 1 %
2 6 1 1 4 3 0 1 1 28 0 1 2 1 1 1 4 3 0 1 1 28 0 1 2 15 1}


\title{R Introductory Session}
\author{J.T. Alcal?}
\date{February 7, 2015}
\maketitle
\tableofcontents
\sloppy


%\abstract{Abstract goes here}




\section{Introduction}

R is a free software environment for statistical computing and graphics.

\url{http://www.r-project.org/}

Manuals and extension packages can be obtained from CRAN at

\url{http://cran.r-project.org/}

\subsection{Getting Help}

Read ``An Introduction to R'' from the CRAN manuals section.

In R online documentation is available with $?$, $??$, $help()$, etc.

For a description of packages that could be interesting for a particular 
task  (e.g., machine learning) 
go to the ``Task Views'' subsection on the CRAN web site.

\section{Basic Data Types}
\subsection{Vector}

\begin{Schunk}
\begin{Sinput}
R> a <- c(1,3,5:10)
R> a
\end{Sinput}
\begin{Soutput}
[1]  1  3  5  6  7  8  9 10
\end{Soutput}
\begin{Sinput}
R> length(a)
\end{Sinput}
\begin{Soutput}
[1] 8
\end{Soutput}
\begin{Sinput}
R> a[1]
\end{Sinput}
\begin{Soutput}
[1] 1
\end{Soutput}
\begin{Sinput}
R> a[-2]
\end{Sinput}
\begin{Soutput}
[1]  1  5  6  7  8  9 10
\end{Soutput}
\begin{Sinput}
R> a[1:3]
\end{Sinput}
\begin{Soutput}
[1] 1 3 5
\end{Soutput}
\begin{Sinput}
R> mean(a)
\end{Sinput}
\begin{Soutput}
[1] 6.125
\end{Soutput}
\begin{Sinput}
R> summary(a)
\end{Sinput}
\begin{Soutput}
   Min. 1st Qu.  Median    Mean 3rd Qu.    Max. 
  1.000   4.500   6.500   6.125   8.250  10.000 
\end{Soutput}
\begin{Sinput}
R> a <- a-1
R> a
\end{Sinput}
\begin{Soutput}
[1] 0 2 4 5 6 7 8 9
\end{Soutput}
\begin{Sinput}
R> names(a) <- 1:length(a)
R> a
\end{Sinput}
\begin{Soutput}
1 2 3 4 5 6 7 8 
0 2 4 5 6 7 8 9 
\end{Soutput}
\begin{Sinput}
R> plot(a)
\end{Sinput}
\end{Schunk}
\includegraphics{irisSweave-002}


\subsection{Matrix}

\begin{Schunk}
\begin{Sinput}
R> m <- matrix(1:25, ncol=5, byrow=FALSE)
R> dim(m)
\end{Sinput}
\begin{Soutput}
[1] 5 5
\end{Soutput}
\begin{Sinput}
R> colnames(m) <- paste("col", 1:ncol(m))
R> rownames(m) <- paste("row", 1:nrow(m))
R> m[1,1:3]
\end{Sinput}
\begin{Soutput}
col 1 col 2 col 3 
    1     6    11 
\end{Soutput}
\begin{Sinput}
R> m[1,]
\end{Sinput}
\begin{Soutput}
col 1 col 2 col 3 col 4 col 5 
    1     6    11    16    21 
\end{Soutput}
\begin{Sinput}
R> m
\end{Sinput}
\begin{Soutput}
      col 1 col 2 col 3 col 4 col 5
row 1     1     6    11    16    21
row 2     2     7    12    17    22
row 3     3     8    13    18    23
row 4     4     9    14    19    24
row 5     5    10    15    20    25
\end{Soutput}
\begin{Sinput}
R> t(m)
\end{Sinput}
\begin{Soutput}
      row 1 row 2 row 3 row 4 row 5
col 1     1     2     3     4     5
col 2     6     7     8     9    10
col 3    11    12    13    14    15
col 4    16    17    18    19    20
col 5    21    22    23    24    25
\end{Soutput}
\begin{Sinput}
R> m*m
\end{Sinput}
\begin{Soutput}
      col 1 col 2 col 3 col 4 col 5
row 1     1    36   121   256   441
row 2     4    49   144   289   484
row 3     9    64   169   324   529
row 4    16    81   196   361   576
row 5    25   100   225   400   625
\end{Soutput}
\begin{Sinput}
R> m%*%m
\end{Sinput}
\begin{Soutput}
      col 1 col 2 col 3 col 4 col 5
row 1   215   490   765  1040  1315
row 2   230   530   830  1130  1430
row 3   245   570   895  1220  1545
row 4   260   610   960  1310  1660
row 5   275   650  1025  1400  1775
\end{Soutput}
\begin{Sinput}
R> colSums(m)
\end{Sinput}
\begin{Soutput}
col 1 col 2 col 3 col 4 col 5 
   15    40    65    90   115 
\end{Soutput}
\begin{Sinput}
R> diag(m)
\end{Sinput}
\begin{Soutput}
[1]  1  7 13 19 25
\end{Soutput}
\end{Schunk}


\begin{Schunk}
\begin{Sinput}
R> image(m)
\end{Sinput}
\end{Schunk}
\includegraphics{irisSweave-004}

3D plot
\begin{Schunk}
\begin{Sinput}
R> persp(m, shade=0.1, theta=30)
\end{Sinput}
\end{Schunk}
\includegraphics{irisSweave-005}

\subsection{List}
\begin{Schunk}
\begin{Sinput}
R> l <- list(a=1:5, b=c("hello", "world"), c=matrix(1:10, ncol=2))
R> l
\end{Sinput}
\begin{Soutput}
$a
[1] 1 2 3 4 5

$b
[1] "hello" "world"

$c
     [,1] [,2]
[1,]    1    6
[2,]    2    7
[3,]    3    8
[4,]    4    9
[5,]    5   10
\end{Soutput}
\begin{Sinput}
R> l[[2]]
\end{Sinput}
\begin{Soutput}
[1] "hello" "world"
\end{Soutput}
\begin{Sinput}
R> l$b
\end{Sinput}
\begin{Soutput}
[1] "hello" "world"
\end{Soutput}
\begin{Sinput}
R> l[["b"]]
\end{Sinput}
\begin{Soutput}
[1] "hello" "world"
\end{Soutput}
\begin{Sinput}
R> str(l)
\end{Sinput}
\begin{Soutput}
List of 3
 $ a: int [1:5] 1 2 3 4 5
 $ b: chr [1:2] "hello" "world"
 $ c: int [1:5, 1:2] 1 2 3 4 5 6 7 8 9 10
\end{Soutput}
\begin{Sinput}
R> rev(l)
\end{Sinput}
\begin{Soutput}
$c
     [,1] [,2]
[1,]    1    6
[2,]    2    7
[3,]    3    8
[4,]    4    9
[5,]    5   10

$b
[1] "hello" "world"

$a
[1] 1 2 3 4 5
\end{Soutput}
\begin{Sinput}
R> lapply(l, rev)
\end{Sinput}
\begin{Soutput}
$a
[1] 5 4 3 2 1

$b
[1] "world" "hello"

$c
 [1] 10  9  8  7  6  5  4  3  2  1
\end{Soutput}
\end{Schunk}

Data.frames are special lists where each element has the same length
\begin{Schunk}
\begin{Sinput}
R> df <- data.frame(number=1:3, letter=c("A","B","C"))
R> df
\end{Sinput}
\begin{Soutput}
  number letter
1      1      A
2      2      B
3      3      C
\end{Soutput}
\begin{Sinput}
R> df$letter
\end{Sinput}
\begin{Soutput}
[1] A B C
Levels: A B C
\end{Soutput}
\end{Schunk}

Typically input data is stored as a data.frame.

\subsection{Coercion Between Data Types}

\begin{Schunk}
\begin{Sinput}
R> m
\end{Sinput}
\begin{Soutput}
      col 1 col 2 col 3 col 4 col 5
row 1     1     6    11    16    21
row 2     2     7    12    17    22
row 3     3     8    13    18    23
row 4     4     9    14    19    24
row 5     5    10    15    20    25
\end{Soutput}
\begin{Sinput}
R> class(m)
\end{Sinput}
\begin{Soutput}
[1] "matrix"
\end{Soutput}
\begin{Sinput}
R> m_df <- as.data.frame(m)
R> m_df
\end{Sinput}
\begin{Soutput}
      col 1 col 2 col 3 col 4 col 5
row 1     1     6    11    16    21
row 2     2     7    12    17    22
row 3     3     8    13    18    23
row 4     4     9    14    19    24
row 5     5    10    15    20    25
\end{Soutput}
\begin{Sinput}
R> class(m_df)
\end{Sinput}
\begin{Soutput}
[1] "data.frame"
\end{Soutput}
\begin{Sinput}
R> as.vector(m)
\end{Sinput}
\begin{Soutput}
 [1]  1  2  3  4  5  6  7  8  9 10 11 12 13 14 15 16 17 18 19 20 21 22
[23] 23 24 25
\end{Soutput}
\begin{Sinput}
R> as.character(m)
\end{Sinput}
\begin{Soutput}
 [1] "1"  "2"  "3"  "4"  "5"  "6"  "7"  "8"  "9"  "10" "11" "12" "13"
[14] "14" "15" "16" "17" "18" "19" "20" "21" "22" "23" "24" "25"
\end{Soutput}
\begin{Sinput}
R> as.logical(m)
\end{Sinput}
\begin{Soutput}
 [1] TRUE TRUE TRUE TRUE TRUE TRUE TRUE TRUE TRUE TRUE TRUE TRUE TRUE
[14] TRUE TRUE TRUE TRUE TRUE TRUE TRUE TRUE TRUE TRUE TRUE TRUE
\end{Soutput}
\end{Schunk}

\section{Reading/Writing Data and Saving Plots}
\subsection{Reading/Writing Data}

try \verb|? read.table| and \verb|? write.table|

\begin{Schunk}
\begin{Sinput}
R> write.csv(m, file="matrix.csv")
R> m2 <- read.csv("matrix.csv")
R> m2
\end{Sinput}
\begin{Soutput}
      X col.1 col.2 col.3 col.4 col.5
1 row 1     1     6    11    16    21
2 row 2     2     7    12    17    22
3 row 3     3     8    13    18    23
4 row 4     4     9    14    19    24
5 row 5     5    10    15    20    25
\end{Soutput}
\end{Schunk}

\subsection{Creating Plots for Documents}

Create a pdf file
\begin{Schunk}
\begin{Sinput}
R> data_rand <- cbind(x=rnorm(100), y=rnorm(100))
R> pdf(file="myplot.pdf")
R> plot(data_rand, xlim=c(-4,4), ylim=c(-4,4))
R> dev.off()
\end{Sinput}
\begin{Soutput}
null device 
          1 
\end{Soutput}
\end{Schunk}

Create a png file
\begin{Schunk}
\begin{Sinput}
R> png(file="myplot.png")
R> plot(data_rand, xlim=c(-4,4), ylim=c(-4,4))
R> dev.off()
\end{Sinput}
\begin{Soutput}
null device 
          1 
\end{Soutput}
\end{Schunk}

\section{Programming Elements}

\subsection{Control Structures}

\begin{Schunk}
\begin{Sinput}
R> a <- c()
R> for(i in 1:10) {a <- append(a, 100+i)}
R> a
\end{Sinput}
\begin{Soutput}
 [1] 101 102 103 104 105 106 107 108 109 110
\end{Soutput}
\end{Schunk}

Better
\begin{Schunk}
\begin{Sinput}
R> a <- numeric(10)
R> for(i in 1:10) {a[i] <- 100+i}
R> a
\end{Sinput}
\begin{Soutput}
 [1] 101 102 103 104 105 106 107 108 109 110
\end{Soutput}
\end{Schunk}

while and if work as expected (see \verb|?Control|)

Comparisons and logical vectors
\begin{Schunk}
\begin{Sinput}
R> a>103
\end{Sinput}
\begin{Soutput}
 [1] FALSE FALSE FALSE  TRUE  TRUE  TRUE  TRUE  TRUE  TRUE  TRUE
\end{Soutput}
\begin{Sinput}
R> which(a>103)
\end{Sinput}
\begin{Soutput}
[1]  4  5  6  7  8  9 10
\end{Soutput}
\end{Schunk}

\begin{Schunk}
\begin{Sinput}
R> a[a>103] <- NA
R> a
\end{Sinput}
\begin{Soutput}
 [1] 101 102 103  NA  NA  NA  NA  NA  NA  NA
\end{Soutput}
\end{Schunk}

Try to prevent loops. Most things in R are vectorized. For example
to add all numbers in a vector which are divisible by 3 together.

\begin{Schunk}
\begin{Sinput}
R> v <- as.integer(runif(10000,1,100))
\end{Sinput}
\end{Schunk}

Don't use a loop like this
\begin{Schunk}
\begin{Sinput}
R> s <- 0
R> system.time(
+ for (i in 1:length(v)) {
+     if(v[i]%%3==0) s <- s+v[i]
+     }
+     )
\end{Sinput}
\begin{Soutput}
   user  system elapsed 
  0.016   0.000   0.016 
\end{Soutput}
\end{Schunk}


But do this
\begin{Schunk}
\begin{Sinput}
R> system.time(s <- sum(v[v%%3==0]))
\end{Sinput}
\begin{Soutput}
   user  system elapsed 
  0.000   0.000   0.001 
\end{Soutput}
\end{Schunk}


\subsection{Functions}

Defining a function
\begin{Schunk}
\begin{Sinput}
R> dec <- function(x) { 
+     x-1 
+ }
R> dec
\end{Sinput}
\begin{Soutput}
function(x) { 
    x-1 
}
\end{Soutput}
\begin{Sinput}
R> dec(1:10)
\end{Sinput}
\begin{Soutput}
 [1] 0 1 2 3 4 5 6 7 8 9
\end{Soutput}
\end{Schunk}

A function returning a function

\begin{Schunk}
\begin{Sinput}
R> gen_dec <- function(i) { function(x) {x-i} }
R> my_dec <- gen_dec(5)
R> my_dec(1:10)
\end{Sinput}
\begin{Soutput}
 [1] -4 -3 -2 -1  0  1  2  3  4  5
\end{Soutput}
\end{Schunk}

\subsection{Object orientation}

R supports two ways of implementing oo called S3 and S4. In S3 a generic
functions can have implementations for different objects and be dispatched
depending on the object they operate on (typically the first argument).  

The name convention for methods is\\
\verb|<function name>.<object type>|

For example see plot. Type \verb|plot.| and hit tab twice. Then type e.g. 
\verb|? plot.default|.

Defining and using a class
\begin{Schunk}
\begin{Sinput}
R> v <- 1:5
R> v
\end{Sinput}
\begin{Soutput}
[1] 1 2 3 4 5
\end{Soutput}
\begin{Sinput}
R> class(v)
\end{Sinput}
\begin{Soutput}
[1] "integer"
\end{Soutput}
\begin{Sinput}
R> class(v) <- "myclass"
R> v
\end{Sinput}
\begin{Soutput}
[1] 1 2 3 4 5
attr(,"class")
[1] "myclass"
\end{Soutput}
\begin{Sinput}
R> attributes(v)
\end{Sinput}
\begin{Soutput}
$class
[1] "myclass"
\end{Soutput}
\begin{Sinput}
R> print.myclass <- function(x) {cat("My class: ", unclass(x), "\n")}
R> v
\end{Sinput}
\begin{Soutput}
My class:  1 2 3 4 5 
\end{Soutput}
\end{Schunk}

S4 provides oo-support with formal class descriptions.

\section{Applications}

\subsection{Inspecting Data}


This famous (Fisher's or Anderson's) iris data set gives the
measurements in centimeters of the variables sepal length and
width and petal length and width, respectively, for 50 flowers
from each of 3 species of iris.  The species are Iris setosa,
     versicolor, and virginica.

% \begin{center}
% \includegraphics[width=5cm]{iris-flower.jpeg}
% \end{center}


\begin{Schunk}
\begin{Sinput}
R> data(iris)
R> head(iris)
\end{Sinput}
\begin{Soutput}
  Sepal.Length Sepal.Width Petal.Length Petal.Width Species
1          5.1         3.5          1.4         0.2  setosa
2          4.9         3.0          1.4         0.2  setosa
3          4.7         3.2          1.3         0.2  setosa
4          4.6         3.1          1.5         0.2  setosa
5          5.0         3.6          1.4         0.2  setosa
6          5.4         3.9          1.7         0.4  setosa
\end{Soutput}
\begin{Sinput}
R> summary(iris)
\end{Sinput}
\begin{Soutput}
  Sepal.Length    Sepal.Width     Petal.Length    Petal.Width   
 Min.   :4.300   Min.   :2.000   Min.   :1.000   Min.   :0.100  
 1st Qu.:5.100   1st Qu.:2.800   1st Qu.:1.600   1st Qu.:0.300  
 Median :5.800   Median :3.000   Median :4.350   Median :1.300  
 Mean   :5.843   Mean   :3.057   Mean   :3.758   Mean   :1.199  
 3rd Qu.:6.400   3rd Qu.:3.300   3rd Qu.:5.100   3rd Qu.:1.800  
 Max.   :7.900   Max.   :4.400   Max.   :6.900   Max.   :2.500  
       Species  
 setosa    :50  
 versicolor:50  
 virginica :50  
\end{Soutput}
\begin{Sinput}
R> data <- iris[, -5]
R> class <- iris[,5]
R> boxplot(data)
\end{Sinput}
\end{Schunk}
\includegraphics{irisSweave-022}

Data can be scaled
\begin{Schunk}
\begin{Sinput}
R> data_s <- scale(data)
R> boxplot(as.data.frame(data_s))
\end{Sinput}
\end{Schunk}
\includegraphics{irisSweave-023}


We can look at the data as a trellis plot.
\begin{Schunk}
\begin{Sinput}
R> plot(data, col=class)
\end{Sinput}
\end{Schunk}
\includegraphics{irisSweave-024}

We can look at a parallel coordinates plot
\begin{Schunk}
\begin{Sinput}
R> library("MASS")
R> parcoord(data, col=class, pch=0)
\end{Sinput}
\end{Schunk}
\includegraphics{irisSweave-025}

\subsection{Linear Regression}

We fit a linear model for sepal length based on petals on part of the data 
(train). 

\begin{Schunk}
\begin{Sinput}
R> train.ind <- sample(1:nrow(iris), 100)
R> train <- iris[train.ind,]
R> test <- iris[-train.ind,]
R> 
\end{Sinput}
\end{Schunk}

lm uses a formula interface (many model building functions in
R support this kind of interface; see \url{http://cran.r-project.org/doc/manuals/R-intro.html#Formulae-for-statistical-models}).

\begin{Schunk}
\begin{Sinput}
R> model <- lm(Sepal.Length ~ Petal.Length + Petal.Width, data = train)
R> model
\end{Sinput}
\begin{Soutput}
Call:
lm(formula = Sepal.Length ~ Petal.Length + Petal.Width, data = train)

Coefficients:
 (Intercept)  Petal.Length   Petal.Width  
      4.1904        0.5402       -0.3408  
\end{Soutput}
\begin{Sinput}
R> plot(model, which=1)
\end{Sinput}
\end{Schunk}
\includegraphics{irisSweave-027}

Use the linear model to predict sepal length in the test data.

\begin{Schunk}
\begin{Sinput}
R> Sepal.Length.predicted <- predict(model, test)
R> plot(Sepal.Length.predicted, test$Sepal.Length)
R> abline(0,1, col="red")
\end{Sinput}
\end{Schunk}
\includegraphics{irisSweave-028}

\subsection{Cluster Analysis}


Clustering with $k$-means.
\begin{Schunk}
\begin{Sinput}
R> cl <- kmeans(data, centers=3)
R> cl
\end{Sinput}
\begin{Soutput}
K-means clustering with 3 clusters of sizes 62, 50, 38

Cluster means:
  Sepal.Length Sepal.Width Petal.Length Petal.Width
1     5.901613    2.748387     4.393548    1.433871
2     5.006000    3.428000     1.462000    0.246000
3     6.850000    3.073684     5.742105    2.071053

Clustering vector:
  [1] 2 2 2 2 2 2 2 2 2 2 2 2 2 2 2 2 2 2 2 2 2 2 2 2 2 2 2 2 2 2 2 2
 [33] 2 2 2 2 2 2 2 2 2 2 2 2 2 2 2 2 2 2 1 1 3 1 1 1 1 1 1 1 1 1 1 1
 [65] 1 1 1 1 1 1 1 1 1 1 1 1 1 3 1 1 1 1 1 1 1 1 1 1 1 1 1 1 1 1 1 1
 [97] 1 1 1 1 3 1 3 3 3 3 1 3 3 3 3 3 3 1 1 3 3 3 3 1 3 1 3 1 3 3 1 1
[129] 3 3 3 3 3 1 3 3 3 3 1 3 3 3 1 3 3 3 1 3 3 1

Within cluster sum of squares by cluster:
[1] 39.82097 15.15100 23.87947
 (between_SS / total_SS =  88.4 %)

Available components:

[1] "cluster"      "centers"      "totss"        "withinss"    
[5] "tot.withinss" "betweenss"    "size"         "iter"        
[9] "ifault"      
\end{Soutput}
\begin{Sinput}
R> table(class, cl$cluster)
\end{Sinput}
\begin{Soutput}
class         1  2  3
  setosa      0 50  0
  versicolor 48  0  2
  virginica  14  0 36
\end{Soutput}
\end{Schunk}

Use PCA for display

\begin{Schunk}
\begin{Sinput}
R> pc <- prcomp(data)
R> pc
\end{Sinput}
\begin{Soutput}
Standard deviations:
[1] 2.0562689 0.4926162 0.2796596 0.1543862

Rotation:
                     PC1         PC2         PC3        PC4
Sepal.Length  0.36138659 -0.65658877  0.58202985  0.3154872
Sepal.Width  -0.08452251 -0.73016143 -0.59791083 -0.3197231
Petal.Length  0.85667061  0.17337266 -0.07623608 -0.4798390
Petal.Width   0.35828920  0.07548102 -0.54583143  0.7536574
\end{Soutput}
\begin{Sinput}
R> summary(pc)
\end{Sinput}
\begin{Soutput}
Importance of components:
                          PC1     PC2    PC3     PC4
Standard deviation     2.0563 0.49262 0.2797 0.15439
Proportion of Variance 0.9246 0.05307 0.0171 0.00521
Cumulative Proportion  0.9246 0.97769 0.9948 1.00000
\end{Soutput}
\begin{Sinput}
R> plot(pc$x, col=class, pch=cl$cluster)
\end{Sinput}
\end{Schunk}
\includegraphics{irisSweave-030}

\subsection{Hierarchical Clustering}

\begin{Schunk}
\begin{Sinput}
R> d <- dist(data)
R> str(d)
\end{Sinput}
\begin{Soutput}
Class 'dist'  atomic [1:11175] 0.539 0.51 0.648 0.141 0.616 ...
  ..- attr(*, "Size")= int 150
  ..- attr(*, "Diag")= logi FALSE
  ..- attr(*, "Upper")= logi FALSE
  ..- attr(*, "method")= chr "euclidean"
  ..- attr(*, "call")= language dist(x = data)
\end{Soutput}
\begin{Sinput}
R> hc <- hclust(d)
R> hc
\end{Sinput}
\begin{Soutput}
Call:
hclust(d = d)

Cluster method   : complete 
Distance         : euclidean 
Number of objects: 150 
\end{Soutput}
\begin{Sinput}
R> plot(hc)
\end{Sinput}
\end{Schunk}
\includegraphics{irisSweave-031}

\begin{Schunk}
\begin{Sinput}
R> d <- dist(data)
R> labels <- cutree(hc,3)
R> table(class,labels)
\end{Sinput}
\begin{Soutput}
            labels
class         1  2  3
  setosa     50  0  0
  versicolor  0 23 27
  virginica   0 49  1
\end{Soutput}
\end{Schunk}

\subsection{Classification}

Here we use the package ``caret'' so we have to install it by typing
\verb|install.packages("caret", depend = TRUE)|.

We use 90\% as the training set and 10\% as the test set.
\begin{Schunk}
\begin{Sinput}
R> train_index <- sample(1:nrow(data), .9*nrow(data), replace = FALSE)
R> train_data <- data[train_index,]
R> train_class <- class[train_index]
R> test_data <- data[-train_index,]
R> test_class <- class[-train_index]
R> library("caret")
R> knn <- knn3(as.matrix(train_data), train_class, k = 5)
R> knn
\end{Sinput}
\begin{Soutput}
5-nearest neighbor classification model

Call:
knn3.matrix(x = as.matrix(train_data), y = train_class, k = 5)

Training set class distribution:

    setosa versicolor  virginica 
        41         46         48 
\end{Soutput}
\begin{Sinput}
R> class(knn)
\end{Sinput}
\begin{Soutput}
[1] "knn3"
\end{Soutput}
\end{Schunk}

Now we can use the model to make predictions for our test set (most model
types in R have a predict method).

\begin{Schunk}
\begin{Sinput}
R> pred <- predict(knn, test_data, type="class")
R> conf_tab <- table(test_class, pred)
R> conf_tab
\end{Sinput}
\begin{Soutput}
            pred
test_class   setosa versicolor virginica
  setosa          9          0         0
  versicolor      0          4         0
  virginica       0          0         2
\end{Soutput}
\begin{Sinput}
R> sum(diag(conf_tab))/sum(conf_tab)
\end{Sinput}
\begin{Soutput}
[1] 1
\end{Soutput}
\end{Schunk}

Since knn is of class knn3 use
\verb|? predict.knn3| to get help for predict for class knn3.

Caret supports also a formula interface 
Let's use caret to search for parameters and do 10-fold cross validation.

\begin{Schunk}
\begin{Sinput}
R> knn_opt <- train(Species~., data=iris, method="knn", 
+     trControl= trainControl(method="cv"),
+     tuneLength=5)
R> knn_opt
\end{Sinput}
\begin{Soutput}
k-Nearest Neighbors 

150 samples
  4 predictors
  3 classes: 'setosa', 'versicolor', 'virginica' 

No pre-processing
Resampling: Cross-Validated (10 fold) 

Summary of sample sizes: 135, 135, 135, 135, 135, 135, ... 

Resampling results across tuning parameters:

  k   Accuracy   Kappa  Accuracy SD  Kappa SD  
   5  0.9666667  0.95   0.03513642   0.05270463
   7  0.9800000  0.97   0.03220306   0.04830459
   9  0.9666667  0.95   0.05665577   0.08498366
  11  0.9866667  0.98   0.02810913   0.04216370
  13  0.9800000  0.97   0.04499657   0.06749486

Accuracy was used to select the optimal model using  the
 largest value.
The final value used for the model was k = 11. 
\end{Soutput}
\end{Schunk}

Only use Petal.Length and Sepal.Length for classification
\begin{Schunk}
\begin{Sinput}
R> knn_opt <- train(Species~Petal.Length+Sepal.Length, data=iris, method="knn", 
+     trControl= trainControl(method="cv"),
+     tuneLength=5)
R> knn_opt
\end{Sinput}
\begin{Soutput}
k-Nearest Neighbors 

150 samples
  4 predictors
  3 classes: 'setosa', 'versicolor', 'virginica' 

No pre-processing
Resampling: Cross-Validated (10 fold) 

Summary of sample sizes: 135, 135, 135, 135, 135, 135, ... 

Resampling results across tuning parameters:

  k   Accuracy   Kappa  Accuracy SD  Kappa SD  
   5  0.9400000  0.91   0.04919099   0.07378648
   7  0.9533333  0.93   0.05488484   0.08232726
   9  0.9466667  0.92   0.06126244   0.09189366
  11  0.9533333  0.93   0.05488484   0.08232726
  13  0.9533333  0.93   0.04499657   0.06749486

Accuracy was used to select the optimal model using  the
 largest value.
The final value used for the model was k = 13. 
\end{Soutput}
\end{Schunk}

For more models try \verb|? train|



\section{Contact Information}

For questions contact me in my office or per email 
at \url{jtalcala@unizar.es}


%
%\bibliographystyle{abbrvnat}
%\bibliography{seriation}
%
\end{document}
